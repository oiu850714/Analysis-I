\section{Real exponentiation, part I} \label{sec 5.6}

In \SEC{4.3} we defined exponentiation \(x^n\) when \(x\) is \emph{rational} and \(n\) is a \emph{natural} number, or when \(x\) is a \emph{non-zero} rational and \(n\) is an \emph{integer}(i.e. can be negative).
Now that we have all the arithmetic operations on the reals
(and \PROP{5.4.7} assures us that the arithmetic properties of the rationals that we are used to, continue to hold for the reals),
we can similarly define \emph{exponentiation} of the \emph{reals}.

\begin{definition} [Exponentiating a real by a \emph{natural} number] \label{def 5.6.1}
Let \(x\) be a real number.
To raise \(x\) to the power \(0\), we define \(x^0 := 1\).
Now suppose recursively(inductively) that \(x^n\) has been defined for some natural number \(n\), then we define \(x^{n + 1} := x^n \X x\).
\end{definition}

\begin{definition} [Exponentiating a real by an (negative) \emph{integer}] \label{def 5.6.2}
Let \(x\) be a \emph{non-zero} real number.
Then for any \emph{negative} integer \(-n\), we define \(x^{-n} := 1/x^n\).
\end{definition}

Clearly these definitions are consistent with the definition of rational exponentiation given earlier (\DEF{4.3.9} and \DEF{4.3.11}).
We can then assert:

\begin{proposition} \label{prop 5.6.3}
All the properties in \PROP{4.3.10} and \PROP{4.3.12} remain valid if \(x\) and \(y\) are assumed to be \emph{real numbers} instead of \emph{rational numbers}.
\end{proposition}

\begin{note}
Instead of giving an actual proof of this proposition, we shall give a \emph{meta-proof} (an argument appealing to \emph{the nature of proofs}, rather than the nature of real and rational numbers).
\end{note}

\begin{meta-proof}
If one inspects the proof of \PROP{4.3.10} and \PROP{4.3.12} we see that they rely on the laws of algebra and the laws of order for the rationals (\PROP{4.2.4} and \PROP{4.2.9}).
But by \PROP{5.3.11}, \PROP{5.4.7}, and the identity \(xx^{-1} = x^{-1}x = 1\)(in the middle of page 110)
we know that all these laws of algebra and order continue to hold for real numbers as well as rationals.
Thus we can modify the proof of \PROP{4.3.10} and \PROP{4.3.12} to hold in the case when \(x\) and \(y\) are \emph{real}.
\end{meta-proof}

Now we consider exponentiation to exponents which are \emph{not} integers.
We begin with the notion of an \(n^{\text{th}}\) root, which we can \emph{define using our notion of supremum}.

\begin{definition} \label{def 5.6.4}.
Let \(x \ge 0\) be a \emph{non-negative} \emph{real}, and let \(n \ge 1\) be a \emph{positive integer}.
We \emph{define} \(x^{1/n}\), also known as the \(n^{\text{th}}\) root of \(x\), by the
formula
\[
    x^{1 / n} := \sup \{y \in \SET{R} : y \ge 0 \text{ and } y^n \le x\}.
\]
We often write \(\sqrt{x}\) for \(x^{1/2}\).
\end{definition}

\begin{note}
Note we do not define the \(n^{\text{th}}\) roots of a \emph{negative} number.
In fact, we will \emph{leave} the \(n^{\text{th}}\) roots of negative numbers \emph{undefined} for the rest of the text
(one can define these \(n^{\text{th}}\) roots once one defines the \emph{complex numbers}, but we shall refrain from doing so).
\end{note}

\begin{lemma} [Existence of nth roots] \label{lem 5.6.5}
Let \(x \ge 0\) be a non-negative real, and let \(n \ge 1\) be a positive integer.
Then the set \(E := \{ y \in \SET{R} : y \ge 0 \text{ and } y^n \le x \} \) is non-empty and is also bounded above. (So by \THM{5.5.9} the \(\sup E\) exits.)
In particular, \(x^{1/n} = \sup E\) is a real number.
\end{lemma}

\begin{proof}
The set \(E\) contains \BLUE{0}, so it is certainly not empty.
(why? Because (1) \(\BLUE{0} \ge 0\) (2) given \emph{positive} integer \(n = m + 1\)(for some natural number \(m\), by \LEM{2.2.10}), by \DEF{5.6.2}, \(\BLUE{0}^n = 0^m \X 0 = \GREEN{0}\), but \(x \ge \GREEN{0}\), so \(\BLUE{0}^n \le x\).
So together we have \(\BLUE{0} \ge 0\) and \(\BLUE{0}^n \le x\), so by construction of \(E\), \(\BLUE{0} \in E\).)

Now we show it has an upper bound.
We divide into two cases: \(x \le 1\) and \(x > 1\).

First suppose that we are in the case where \(x \le 1\).
Then we claim that the set \(E\) is bounded above by \(1\).
To see this, suppose for sake of contradiction that there was an element \(y \in E\) for which \(y > 1\). But then \(y^n > 1\)
(why? It can be proved by induction that given \(y > 1\), \(y^m > 1\) for all natural number \(m\).
So in particular for positive integer(i.e. natural number) \(n\), \(y^n > 1\)),
and hence \(y^n > x\), so by construction of \(E\), \(y \notin E\), a contradiction.
Thus \(E\) has an upper bound \(1\).

Now suppose that we are in the case where  \(x > 1\). Then we claim that the set \(E\) is bounded above by \(x\).
To see this, suppose for contradiction that there was an element \(y \in E\) for which \(y > x\).
Since \(x > 1\), we thus have \(y > 1\). Since \(y > x\) and \(y > 1\), we have \(y^n > x\)
(why? Again it can be proved by induction that given \(y > x\), \(y^m > x\) for all natural number \(m\).
So in particular for positive integer(i.e. natural number) \(n\), \(y^n > x\)),
so by construction of \(E\), \(y \notin E\), a contradiction.
Thus \(E\) has an upper bound \(x\).

Thus in both cases \(E\) has an upper bound, hence \(x^{1/n} = \sup E\) exists.
\end{proof}

We list some basic properties of \(n^{\text{th}}\) roots below.

\begin{lemma} \label{lem 5.6.6}
Let \(x, y \ge 0\) be non-negative reals, and let \(n, m \ge 1\) be \emph{positive} integers.
\begin{enumerate}
    \item If \(y = x^{1 / n}\), then \(y^n = x\), so \(x = y^n = (x^{1/n})^n\).
    \item Conversely, if \(y^n = x\), then \(y = x^{1 / n}\), so \(y = x^{1/n} = (y^n)^{1/n}\).
    \item \(x^{1 / n}\) is a non-negative real number, and is positive if and only if \(x\) is positive.
    \item We have \(x > y\) if and only if \(x^{1 / n} > y^{1 / n}\).
    \item Let \(k\) be a positive integer.
        If \(x > 1\), then \(x^{1 / k}\) is a (strictly) \emph{decreasing} function of \(k\).
        If \(0 < x < 1\), then \(x^{1 / k}\) is an (strictly) \emph{increasing} function of \(k\).
        If \(x = 1\), then \(x^{1 / k} = 1\) for all \(k\).
    \item We have \((xy)^{1 / n} = x^{1 / n}y^{1 / n}\).
    \item We have \((x^{1 / n})^{1 / m} = x^{1 / nm}\).
\end{enumerate}
\end{lemma}

\begin{note}
The part (c) of the lemma is different from the textbook, which seems wrong.
For part(a), the point is to make sure the \DEF{5.6.4} is really what we expect.
That is, not only the \(n^{\text{th}}\) root of \(x\) is a real number, it also satisfies that the \(n^{\text{th}}\) power of the \(n^{\text{th}}\) root of \(x\) is actually equal to \(x\).
\end{note}

\begin{note}
(a) 先開\ \(n\) 次根再取\ \(n\) 次方會等於自己。
(b) 先取\ \(n\) 次方再開\ \(n\) 次根後會等於自己。
(c) 非負實數開幾次根都還是非負;而正實數根幾次根都是正數。
(d) 若\ \(x > y\),則\ \(x\) 開\ \(n\) 次根大於\ \(y\) 開\ \(n\) 次根。
(f) 先相乘再開\ \(n\) 次根等於先各自開\ \(n\) 次根再相乘。
(g) 先開\ \(n\) 次根再開\ \(m\) 次根等於直接開\ \(n \X m\) 次根。
\end{note}

\begin{proof}
Let \(n\) be arbitrary positive integer, and \(x, y\) be non-negative real.
\begin{enumerate}
\item
    Suppose \(y = x^{1 / n} = \sup E\) where \(E := \{ z \in \SET{R}: z \ge 0 \land z^n \le x \}) \).
    We have to show \(y^n = x\).
    Like \PROP{5.5.12}, we will show that both \(y^n < x\) and \(y^n > x\) lead to contradictions.
    \begin{itemize}
    \item [\(y^n < x\)]:
        Then we first show that given arbitrary positive integer \(n\), and given arbitrary non-negative real number \(y\), there exists a positive real number \(M\) s.t.
        \emph{for all real} \(0 < \delta < 1\), \((y + \delta)^n \le y^n + \delta M\) \MAROON{(*)}.
        
        We prove this by induction with the base case \(n = 1\).
        
        For the base case \(n = 1\), given arbitrary \(y \ge 0\); then for all \(0 < \delta < 1\),
        \begin{align*}
        (y + \delta)^1 & = y + \delta & \text{from \DEF{5.6.1}} \\
                       & = y^1 + \delta & \text{from \DEF{5.6.1}} \\
                       & = y^1 + 1 \X \delta, & \text{by \PROP{4.2.4}(7), with \(\delta\) as real number} \\
                       & \le y^1 + 1 \X \delta, & \text{in particular}
        \end{align*}
        so we can find \(M = 1\) as required (BTW this derivation in fact does not depend on the value of \(\delta\)).
        
        Suppose inductively that for some integer \(n \ge 1\), given arbitrary \(y \ge 0\), we can find some positive real \(M\) s.t. for all \(0 < \delta < 1\), \((y + \delta)^n \le y^n + M\delta\);
        we have to find another positive real \(M'\) s.t. for all \(0 < \delta < 1\), \((y + \delta)^{n + 1} \le y^{n + 1} + M' \delta \).
        But
        \begin{align*}
            (y + \delta)^{n + 1} & = (y + \delta)(y + \delta)^n & \text{of course} \\
                                 & \le (y + \delta)(y^n + M\delta) & \text{by inductive hypothesis} \\
                                 & = y^{n + 1} + \delta y^n + yM\delta + M\delta^2 & \text{expand by algebra} \\
                                 & \le y^{n + 1} + \delta y^n + yM\delta + M\delta & \text{since \(0 < \delta < 1\), \(\delta^2 < \delta\)} \\
                                 & = y^{n + 1} + (y^n + yM + M)\delta & \text{of course}
        \end{align*}
        Now let \(M' = y^n + yM + M\), then of course \(M'\) is positive, and for all \(0 < \delta < 1\) we have the inequalities as desired.
        So this closes the induction.
        
        So, from the induction \MAROON{(*)}, given positive integer \(n\) and non-negative real \(y\), we can get that \(M\);
        and once we find \(M\), we want to find a particular \(\varE\) s.t. \(0 < \varE < 1\) and:
        \begin{align*}
                     & y^n < x \\
            \implies & 0 < x - y^n \\
            \implies & 0 < (x - y^n)/M \\
            \implies & 0 < \varE < (x - y^n)/M & \text{for some \(\varE\) by \PROP{5.4.14} or \EXEC{5.5.5}} \\
            \implies & 0 < \varE M < x - y^n \\
            \implies & y^n < y^n + \varE M < x.
        \end{align*}
        So we can found that \(1 > \varE > 0\) s.t. \(y^n + \varE M < x\) \MAROON{(1)}.
        So in particular, \(0 < \varE < 1\), and from the induction \MAROON{(*)} we have \((y + \varE)^n \le y^n + \varE M\), and with \MAROON{(1)} we conclude that \((y + \varE)^n < x\).
        So by construction of \(E\), \(y + \varE \in E\), but that contradicts \(y = x^{1/n} = \sup E\) is the least upper bound of \(E\).
    \item [\(y^n > x\)]:
        Then similarly as previous case, we first show that given arbitrary positive integer \(n\), and given arbitrary non-negative real number \(y\), there exists a positive real number \(M\) s.t.
        \emph{for all real} \(0 < \delta < 1\), \((y - \delta)^n \ge y^n - \delta M\) \MAROON{(**)}.
        
        We prove this by induction with the base case \(n = 1\).
        
        For the base case \(n = 1\), given arbitrary \(y \ge 0\); then for all \(0 < \delta < 1\),
        \begin{align*}
        (y - \delta)^1 & = y - \delta & \text{of course} \\
                       & = y^1 - \delta & \text{of course} \\
                       & = y^1 - 1 \X \delta, & \text{of course!!} \\
                       & \ge y^1 - 1 \X \delta, & \text{in particular!}
        \end{align*}
        so we can find \(M = 1\) as required (BTW this derivation in fact does not depend on the value of \(\delta\)).
        
        Suppose inductively that for some integer \(n \ge 1\), given arbitrary \(y \ge 0\), we can find some positive real \(M\) s.t. for all \(0 < \delta < 1\), \((y - \delta)^n \ge y^n - M\delta\);
        we have to find another positive real \(M'\) s.t. for all \(0 < \delta < 1\), \((y - \delta)^{n + 1} \ge y^{n + 1} - M' \delta \).
        But
        \begin{align*}
            (y - \delta)^{n + 1} & = (y - \delta)(y - \delta)^n & \text{of course} \\
                                 & \ge (y - \delta)(y^n - M\delta) & \text{by inductive hypothesis} \\
                                 & = y^{n + 1} - \delta y^n - yM\delta + M\delta^2 & \text{expand by algebra} \\
                                 & \ge y^{n + 1} - \delta y^n - yM\delta & \text{since \(M\delta^2 > 0\)} \\
                                 & = y^{n + 1} - (y^n + yM)\delta & \text{of course}
        \end{align*}
        Now let \(M' = y^n + yM\), then of course \(M'\) is positive, and for all \(0 < \delta < 1\) we have the inequalities as desired.
        So this closes the induction.
        
        So, from the induction \MAROON{(**)}, given positive integer \(n\) and non-negative real \(y\), we can get that \(M\);
        and once we find \(M\), we want to find a particular \(\varE = -\varE'\) s.t. \(0 > \varE' > -1\) (hence \(0 < \varE < 1\)) and:
        \begin{align*}
                     & y^n > x \\
            \implies & 0 > x - y^n \\
            \implies & 0 > (x - y^n)/M \\
            \implies & 0 > \varE' > (x - y^n)/M & \text{for some \(\varE'\) by \PROP{5.4.14} or \EXEC{5.5.5}} \\
            \implies & 0 > -\varE > (x - y^n)/M \\
            \implies & 0 > -\varE M > x - y^n \\
            \implies & y^n > y^n - \varE M > x.
        \end{align*}
        So we can found that \(1 > \varE > 0\) s.t. \(y^n - \varE M > x\) \MAROON{(2)}.
        So in particular, \(0 < \varE < 1\), and from the induction \MAROON{(**)} we have \((y - \varE)^n \ge y^n - \varE M\), and with \MAROON{(2)} we conclude that \((y - \varE)^n > x\).
        So by construction of \(E\), that implies \((y - \varE)^n\) is an upper bound of \(E\)(since \(x\) is an upper bound of \(E\));
        but that contradicts \(y = x^{1/n} = \sup E\) is the \emph{least} upper bound of \(E\).
    \end{itemize}
    So both \(y^n < x\) and \(y^n > x\) lead to contradictions, so we have \(y^n = x\).
\item
    Suppose \(x, y, n\) satisfy the given condition.
    Now suppose \(y^n = x\).
    For the sake of contradiction, suppose \(y \neq x^{1/n}\). We will show both \(y < x^{1/n}\) and \(y > x^{1/n}\) lead to contradictions.
    
    \begin{itemize}
        \item[\(y < x^{1/n}\)]:
            Then
            \begin{align*}
                         & y < x^{1/n} \\
                \implies & y^n < (x^{1/n})^n & \text{\(n\)'s positive int, with \PROP{4.3.10}(c)} \\
                \implies & y^n < x & \text{by part (a)}
            \end{align*}
            which is a contradiction because \(y^n = x\).
        \item[\(y > x^{1/n}\)]:
            Then
            \begin{align*}
                         & y > x^{1/n} \\
                \implies & y^n > (x^{1/n})^n & \text{\(n\)'s positive int, with \PROP{4.3.10}(c)} \\
                \implies & y^n > x & \text{by part (a)}
            \end{align*}
            which is a contradiction because \(y^n = x\).
    \end{itemize}
\item
    By definition, \(x^{1/n} = \sup E\) where \(E = \{ y \in \SET{R}: y \ge 0 \land y^n \le x \}\).
    But in the proof of \LEM{5.6.5}, we have concluded that \(E\) contains \(0\), so the least upper bound of \(E\) must be greater than or equal to \(0\).
    That is, \(x^{1/n} \ge 0\).

    Now suppose \(x^{1/n}\) is positive.
    Then by \PROP{4.3.10}(c), we can conclude \((x^{1/n})^{n} > 0\).
    But by part(a), we know \((x^{1/n})^n = x\), so \(x > 0\).

    Now suppose \(x > 0\).
    Suppose for the sake of contradiction that \(x^{1/n}\) is not greater than \(0\).
    Since we have shown that \(x^{1/n} \ge 0\), so \(x^{1/n} = 0\).
    But by \PROP{4.3.10}(b), since \(x^{1/n} = 0\), we have \((x^{1/n})^n = 0\).
    By part(a), that implies \(x = (x^{1/n})^n = 0\), a contradiction.
\item
    Suppose \(x > y\) \MAROON{(3)}.
    Suppose for the sake of contradiction that \(x^{1/n} \le y^{1/n}\).
    If \(x^{1/n} \le y^{1/n}\), then from part(c) we have shown that both \(x^{1/n}, y^{1/n}\) are non-negative, so we have \(0 \le x^{1/n} \le y^{1/n}\).
    From \PROP{4.3.10}(c), we have \(0 \le (x^{1/n})^n \le (y^{1/n})^n\).
    By part(a), we have \(0 \le x \le y\), contradicting \MAROON{(3)}.
    So \(x^{1/n} > y^{1/n}\).

    Suppose \(x^{1/n} > y^{1/n}\).
    Then again by part(c), we have \(x^{1/n} > y^{1/n} \ge 0\).
    Also, with \PROP{4.3.10}(c), we have \((x^{1/n})^n > (y^{1/n})^n \ge 0\), in particular  \((x^{1/n})^n > (y^{1/n})^n\).
    From part(a), we have \(x > y\).
\item
    \begin{itemize}
    \item[\(x > 1\)]:
        We have to show \(x^{1/k}\) is a strictly decreasing function of positive integer \(k\).
        That is, we have to show given arbitrary positive integer \(k_1 > k_2\), we have \(x^{1/k_1} < x^{1/k_2}\).
        Suppose for the sake of contradiction that there exist positive integers \(k_1 > k_2\) but \(x^{1/k_1} \ge x^{1/k_2}\).
        By part(c), \(x^{1/k_1} \ge x^{1/k_2} \ge 0\). And
        \begin{align*}
                     & x^{1/k_1} \ge x^{1/k_2} \ge 0 \\
            \implies & (x^{1/k_1})^{k_1} \ge (x^{1/k_2})^{k_1} \ge 0 & \text{by \PROP{4.3.10}(c)} \\
            \implies & x \ge (x^{1/k_2})^{k_1} \ge 0 & \text{by part(a)} \\
            \implies & x^{k_2} \ge ((x^{1/k_2})^{k_1})^{k_2} \ge 0 & \text{by \PROP{4.3.10}(c)} \\
            \implies & x^{k_2} \ge (x^{1/k_2})^{k_1 k_2} \ge 0 & \text{by \PROP{4.3.10}(a)} \\
            \implies & x^{k_2} \ge (x^{1/k_2})^{k_2 k_1} \ge 0 & \text{of course} \\
            \implies & x^{k_2} \ge ((x^{1/k_2})^{k_2})^{k_1} \ge 0 & \text{by \PROP{4.3.10}(a)} \\
            \implies & x^{k_2} \ge x^{k_1} \ge 0 \MAROON{(4)} & \text{by part(a)}
        \end{align*}
        But \(k_1 > k_2\), so \(k_1 = k_2 + r\) for some positive integer \(r\).
        And
        \begin{align*}
            x^{k_1} & = x^{k_2 + r} \\
                    & = x^{k_2} x^{r} & \text{by \PROP{4.3.10}(a)} \\
                    & > x^{k_2} 1 & \text{\(x > 1 \land \text{ (integer) } r > 1 \implies x^r > 1\)} \\
                    & = x^{k_2}
        \end{align*}
        which contradicts \MAROON{(4)} that \(x^{k_2} \ge x^{k_1}\).
        So for \(x > 1\), \(x^{1/k}\) is an strictly decreasing function of positive integer \(k\).
    \item[\(0 < x < 1\)]:
        We have to show \(x^{1/k}\) is a strictly increasing function of positive integer \(k\).
        That is, we have to show given arbitrary positive integer \(k_1 > k_2\), we have \(x^{1/k_1} > x^{1/k_2}\).
        Suppose for the contradiction that there exist positive integers \(k_1 > k_2\) but \(x^{1/k_1} \le x^{1/k_2}\).
        By part(c), \(0 \le x^{1/k_1} \le x^{1/k_2}\). And
        \begin{align*}
                     & 0 \le x^{1/k_1} \le x^{1/k_2} \\
            \implies & 0 \le (x^{1/k_1})^{k_1} \le (x^{1/k_2})^{k_1} & \text{by \PROP{4.3.10}(c)} \\
            \implies & 0 \le x \le (x^{1/k_2})^{k_1} & \text{by part(a)} \\
            \implies & 0 \le x^{k_2} \le ((x^{1/k_2})^{k_1})^{k_2} & \text{by \PROP{4.3.10}(c)} \\
            \implies & 0 \le x^{k_2} \le (x^{1/k_2})^{k_1 k_2} & \text{by \PROP{4.3.10}(a)} \\
            \implies & 0 \le x^{k_2} \le (x^{1/k_2})^{k_2 k_1} & \text{of course} \\
            \implies & 0 \le x^{k_2} \le ((x^{1/k_2})^{k_2})^{k_1} & \text{by \PROP{4.3.10}(a)} \\
            \implies & 0 \le x^{k_2} \le x^{k_1} \MAROON{(5)} & \text{by part(a)}
        \end{align*}
        But \(k_1 > k_2\), so \(k_1 = k_2 + r\) for some positive integer \(r\).
        And
        \begin{align*}
            x^{k_1} & = x^{k_2 + r} \\
                    & = x^{k_2} x^{r} & \text{by \PROP{4.3.10}(a)} \\
                    & < x^{k_2} 1 & \text{\(0 < x < 1 \land \text{ (integer) } r > 1 \implies x^r < 1\)} \\
                    & = x^{k_2}
        \end{align*}
        which contradicts \MAROON{(5)} that \(x^{k_2} \le x^{k_1}\).
        So for \(0 < x < 1\), \(x^{1/k}\) is an strictly increasing function of positive integer \(k\).
    \item[\(x = 1\)]:
        We have to show \(x^{1/k} = 1\) for all positive integer \(k\).
        Suppose for the sake of contradiction there exists a positive integer \(k\) s.t. \(x^{1/k} \neq 1\). Then either \(x^{1/k} > 1\) or \(x^{1/k} < 1\).
        \begin{itemize}
            \item[\(x^{1/k} > 1\)]:
                In particular \(x^{1/k} > 1 \ge 0\).
                Then
                \begin{align*}
                             & x^{1/k} > 1 \ge 0 \\
                    \implies & (x^{1/k})^k > 1^k > 0 & \text{by \PROP{4.3.10}(c)} \\
                    \implies & (x^{1/k})^k > 1 > 0 & \text{of course} \\
                    \implies & x > 1 > 0, & \text{by part(a)}
                \end{align*}
                which contradicts that \(x = 1\).
            \item[\(x^{1/k} < 1\)]:
                Again by part(c), \(0 \le x^{1/k} < 1\). And
                \begin{align*}
                             & 0 \le x^{1/k} < 1 \\
                    \implies & 0 \le (x^{1/k})^k < 1^k & \text{by \PROP{4.3.10}(c)} \\
                    \implies & 0 \le (x^{1/k})^k < 1 & \text{of course} \\
                    \implies & 0 \le x < 1, & \text{by part(a)}
                \end{align*}
                which contradicts that \(x = 1\).
        \end{itemize}
        So \(x^{1/k} = 1\) for all positive integer \(k\).
    \end{itemize}
\item
    By part(a), we have \(((xy)^{1/n})^n = xy\).
    And
    \begin{align*}
        (x^{1/n}y^{1/n})^n & = (x^{1/n})^n (y^{1/n})^n & \text{by \PROP{4.3.10}(a)} \\
                           & = x (y^{1/n})^n & \text{by part(a)} \\
                           & = x y & \text{by part(a)}
    \end{align*}
    So we have \(((xy)^{1/n})^n = (x^{1/n}y^{1/n})^n\),
    and with part(b), we have \((xy)^{1/n} = ((x^{1/n}y^{1/n})^n)^{1/n}\).
    But again by part(b), RHS is equal to \(x^{1/n}y^{1/n}\).
    So we have \((xy)^{1/n} = x^{1/n}y^{1/n}\), as desired.
\item
    \begin{align*}
        ((x^{1/n})^{1/m})^{nm} & = ((x^{1/n})^{1/m})^{mn} & \text{of course} \\
                               & = (((x^{1/n})^{1/m})^m)^n & \text{by \PROP{4.3.10}(a)} \\
                               & = (x^{1/n})^n & \text{by part(a), \(((x^{1/n})^{1/m})^m = x^{1/n}\)} \\
                               & = x & \text{by part(a)}
    \end{align*}
    And by part(a), \((x^{1/mn})^{mn} = x\).
    So together we have \((x^{1/n})^{1/m})^{nm} = (x^{1/mn})^{mn}\),
    and with part(b), we have \((x^{1/n})^{1/m} = ((x^{1/mn})^{mn})^{1/mn}\).
    But again by part(b), RHS is equal to \(x^{1/mn}\).
    So we have \((x^{1/n})^{1/m} = x^{1/mn}\), as desired.
\end{enumerate}
\end{proof}

\begin{note}
The observant reader may note that this definition of \(x^{1/n}\) might possibly be inconsistent with our previous notion of \(x^n\) when \(n = 1\).
The point is that by \DEF{5.6.1}, we have defined what \(x^1\) means.
But since \(1 = 1/1\), we can also define \(x^{1/1}\) using \DEF{5.6.4}.
So we have to check these two definitions give the same real number.
But it is easy to check since they both give the real number \(x\)
(why? \(x^1\) by \DEF{5.6.1} is trivially equal to \(x\).
For \(x^{1/1}\), for \(n = 1\), by \LEM{5.6.6}(a), \((x^{1/n})^n = x\), that is, \((x^{1/1})^1 = x\), but again by \DEF{5.6.1} \((x^{1/1})^1\) is equal to \(x^{1/1}\), so we have \(x^{1/1} = x\).
So both \(x^1\) and \(x^{1/1}\) equal to the real number \(x\) therefore are equal to each other.),
so there is no inconsistency.
\end{note}

\begin{note}
One consequence of \LEM{5.6.6}(b) is the following cancellation law:
if \(y\) and \(z\) are positive and \(y^n = z^n\), then \(y = z\). (Why? From \LEM{5.6.6}(b), we have \(y = (z^n)^{1/n}\). But also from \LEM{5.6.6}(b), we also have \(z = (z^n)^{1/n}\), so \(y = z\).)
Note that this only works when \(y\) and \(z\) are positive;
for instance, \((-3)^2 = 3^2\), but we cannot conclude from this that \(-3 = 3\).
\end{note}

Now we define how to raise a positive number \(x\) to a \emph{rational} exponent \(q\).

\begin{definition} \label{def 5.6.7}
Let \(x > 0\) be a positive real number, and let \(q\) be a rational number.
To define \(x^q\), we write \(q = a/b\) for some integer \(a\) and \emph{positive} integer \(b\), and define
\[
    x^q := (x^{1/b})^a.
\]
\end{definition}

\begin{note}
Note that every rational \(q\), whether positive, negative, or zero, can be written in the form \(a/b\) where \(a\) is an integer and \(b\) is \emph{positive} (why? This is a direct consequence of \DEF{4.2.6}).
However, the rational number \(q\) can be expressed in the form \(a/b\) in more than one way, for instance \(1/2\) can also be expressed as \(2/4\) or \(3/6\).
So to ensure that this definition is well-defined, \emph{we need to check that different expressions \(a/b\) give the same formula for \(x^q\)}:
\end{note}

\begin{note}
意思就是若\ \(a/b = a'/b'\),則根據 \DEF{5.6.7},我們要確定先開\ \(b\) 次根再取\ \(a\) 次方,跟先開\ \(b'\) 次根再取\ \(a'\) 次方,得到的結果要是一樣的。
\end{note}

\begin{lemma} \label{lem 5.6.8}
Let \(a, a'\) be integers and \(b, b'\) be positive integers such that \(a/b = a'/b'\),
and let \(x\) be a positive real number.
Then we have \((x^{1/b'})^{a'} = (x^{1/b})^a\) (so that by \DEF{5.6.7} \(x^{a/b} = x^{a'/b'}\)).
\end{lemma}

\begin{proof}
There are three cases: \(a = 0, a > 0, a < 0\).
If \(a = 0\), then we must have \(a' = 0\)
(why? By \DEF{4.2.1}, \(a/b = a'/b'\) iff \(ab' = a'b\); but LHS = \(0 \X b' = 0\), so RHS \(a'b = 0\); but \(b\) is positive, by \PROP{4.1.8}, we have \(a' = 0\).)
and so both \((x^{1/b'})^{a'} = (x^{1/b'})^0 = 1\) and \((x^{1/b})^a = (x^{1/b})^0 = 1\), so we are done.

Now suppose that \(a > 0\).
Then \(a' > 0\)
(why? Again by \DEF{4.2.1} we have \(ab' = a'b\). Since both \(a, b' > 0\), \(ab' > 0\), so LHS is positive, so RHS \(a'b\) is positive;
and \(b\) also positive, so \(a'\) must be positive.
We can use \AC{4.2.5} and \AC{4.2.6} in each step, although they are for rationals, but integers are also rationals.),
and by \DEF{4.2.1}, we have \(ab' = a'b\), which is equal to \(ba'\).
So write \(y := x^{1/(ab')} = x^{1/(ba')}\).
By \LEM{5.6.6}(g) we have \(y = (x^{1/b'})^{1/a}\) and \(y = (x^{1/b})^{1/a'}\);
by \LEM{5.6.6}(a) we thus have \(y^a = x^{1/b'}\) \MAROON{(1)} and \(y^{a'} = x^{1/b}\) \MAROON{(2)}.
Thus we have
\begin{align*}
    (x^{1/b'})^{a'} & = (y^a)^{a'} & \text{by \MAROON{(1)}} \\
                    & = y^{aa'} & \text{by \PROP{4.3.10}(a)} \\
                    & = (y^{a'})^a & \text{by \PROP{4.3.10}(a)} \\
                    & = (x^{1/b})^a & \text{by \MAROON{(2)}}
\end{align*}
as desired.

Finally, suppose that \(a < 0\). Then we have \((-a)/b = (-a')/b\).
But \(-a\) is positive, so the previous case applies and we have \((x^{1/b'})^{-a'} = (x^{1/b})^{-a}\).
And thus \(((x^{1/b'})^{-a'})^{-1} = ((x^{1/b})^{-a})^{-1}\).
By \PROP{4.3.12}(a), we have \((x^{1/b'})^{-a' \X -1} = (x^{1/b})^{-a \X -1}\), that is, \((x^{1/b'})^{a'} = (x^{1/b})^{a}\), as desired.
\end{proof}

Thus \(x^q\) is well-defined for every rational \(q\).
Note that this new definition is consistent with our old definition(\DEF{5.6.4}) for \(x^{1/n}\)
(why? since
\begin{align*}
    x^{\MAROON{1}/\BLUE{n}} & = (x^{1/\BLUE{n}})^{\MAROON{1}} & \text{by \DEF{5.6.7}} \\
                            & = x^{1/n} & \text{of course}
\end{align*}
)
and is also consistent with our old definition(\DEF{5.6.1}) for \(x^n\)
(why? since
\begin{align*}
    x^n & = x^{\MAROON{n}/\BLUE{1}} & \text{of course} \\
        & = (x^{1/\BLUE{1}})^{\MAROON{n}} & \text{by \DEF{5.6.7}} \\
        & = x^{\MAROON{n}} & \text{we have checked the consistency below the page \(123\) that \(x^{1/1} = x^1 = x\)}
\end{align*}
).

Some basic facts about rational exponentiation:

\begin{lemma} \label{lem 5.6.9}
Let \(x, y > 0\) be positive reals, and let \(q, r\) be rationals.
\begin{enumerate}
    \item \(x^q\) is a \emph{positive} real.
    \item \(x^{q+r}\ = x^q x^r\) and \((x^q)^r = x^{qr}\).
    \item \(x^{-q} = 1/x^q\).
    \item If \(q > 0\), then \(x > y\) if and only if \(x^q > y^q\).
    \item If \(x > 1\), then \(x^q > x^r\) if and only if \(q > r\). If \(x < 1\), then \(x^q > x^r\) if and only if \(q < r\).
    \item \((xy)^q = x^q y^q\).
\end{enumerate}
\end{lemma}

\begin{proof}
\begin{enumerate}
\item
    Let arbitrary \(q = a/b\) where \(a\) is integer and \(b\) is positive integer.
    Then
    \begin{align*}
        x^q & = x^{a/b} \\
            & = (x^{1/b})^a & \text{by \DEF{5.6.7}}
    \end{align*}
    Since \(x\) is positive and \(b\) is positive integer, by \LEM{5.6.6}(c), \(x^{1/b}\) is positive.
    And since \(x^{1/b}\) is positive and \(a\) is just an integer, by \PROP{5.6.3}(or precisely, \PROP{4.3.12}(b)), \((x^{1/b})^a\) is still positive.
    So \(x^q\) is positive for all rationals \(q\).
\item
    Let \(q = a / b\) and \(r = c / d\) where \(a, c\) are integers and \(b, d\) are positive integers.
    Then
    \begin{align*}
        x^{q + r} & = x^{a/b + c/d} \\
                  & = x^{(ad + bc)/bd} & \text{by \DEF{4.2.2}} \\
                  & = (x^{1/bd})^{ad + bc} & \text{by \DEF{5.6.7}} \\
                  & = (x^{1/bd})^{ad} (x^{1/bd})^{bc} & \text{by \PROP{4.3.12}(a)} \\
                  & = x^{ad/bd} x^{bc/bd} & \text{by \DEF{5.6.7}} \\
                  & = x^{a/b} x^{c/d} & \text{\(ad/bd = a/b, bc/bd = c/d\), and \LEM{5.6.8}} \\
                  & = x^q x^r
    \end{align*}

    For the equation \(x^{qr} = (x^q)^r\), we first show that \(((x^q)^r)^{bd} = (x^{qr})^{bd} = x^{ac}\):
    \begin{align*}
        ((x^q)^r)^{bd} & = ((x^{a/b})^{c/d})^{bd} \\
                       & = ((((x^{1/b})^a)^{1/d})^c)^{bd} & \text{by \DEF{5.6.7}} \\
                       & = (((x^{1/b})^a)^{1/d})^{c \X bd} & \text{\(c, bd\) are int, with \PROP{4.3.12}(a)} \\
                       & = ((\BLUE{((x^{1/b})^a)}^{1/d})^d)^{bc} & \text{again by \PROP{4.3.12}(a)} \\
                       & = ((x^{1/b})^a)^{bc} & \text{\(d\)'s positive int, with \LEM{5.6.6}(a)} \\
                       & = ((\BLUE{x}^{1/b})^b)^{ac} & \text{\(a, b, c\) int, apply \PROP{4.3.12}(a) several times} \\
                       & = x^{ac} & \text{\(b\)'s positive int, with \LEM{5.6.6}(a)}
    \end{align*}
    and
    \begin{align*}
        (x^{qr})^{bd} & = (x^{ac/bd})^{bd} \\
                      & = ((x^{1/(bd)})^{ac})^{bd} & \text{by \DEF{5.6.7}} \\
                      & = ((x^{1/(bd)})^{bd})^{ac} & \text{\(ac, bd\) are int, apply \PROP{4.3.12}(a) two times} \\
                      & = x^{ac} & \text{by \LEM{5.6.6}(a)}
    \end{align*}
    Thus \(((x^q)^r)^{bd} = (x^{qr})^{bd}\).
    Since \((x^q)^r\) and \(x^{qr}\) are positive by part(a), and \(bd\) is integer not equal to zero, by \PROP{4.3.12}(c) we have \((x^q)^r = x^{qr}\), as desired.
\item
    We first show that \(x^{-n} = 1/x^n\) \BLUE{(*)} for an integer \(n\), \textbf{regardless of the sign} of \(n\).
    If \(n\) is positive, then this follows by \DEF{5.6.2}.
    If \(n = 0\), then \(-n = 0\) so both sides of \BLUE{(*)} are equal to \(1\).
    If \(n\) is negative, then \(n = -m\) for some positive integer \(m\).
    We have \(x^{-n} = x^{m}\) and
    \begin{align*}
        1/x^n & = 1/x^{-m} \\
              & = 1/(1/x^m) & \text{by previous case that \(m\) is positive} \\
              & = x^m & \text{need to prove but trivial} \\
              & = x^{-n}.
    \end{align*}
    So again \BLUE{(*)} is satisfied.

    Now let \(q = a/b\) for some integer \(a\) and positive integer \(b\).
    Then
    \begin{align*}
        x^{-q} & = x^{-(a/b)} \\
               & = x^{(-a)/b} & \text{by \AC{4.2.3}} \\
               & = (x^{1/b})^{-a} & \text{by \DEF{5.6.7}} \\
               & = 1/(x^{1/b})^{a} & \text{by \BLUE{(*)}} \\
               & = 1/x^q & \text{by \DEF{5.6.7}}
    \end{align*}
\item
    Suppose rational \(q = a/b > 0\) for some \emph{positive} integer \(a, b\).
    \begin{itemize}
    \item[\(\Longrightarrow\)]
        \begin{align*}
                     & x > y \\
            \implies & x^{1/b} > y^{1/b} & \text{\(b\)'s positive int,} \\
                     &                   & \text{by \LEM{5.6.6}(d)'s \(\Longrightarrow\) direction} \\
            \implies & (x^{1/b})^a > (y^{1/b})^a & \text{\(a\)'s positive int, by \PROP{4.3.10}(c)} \\
            \implies & x^q > y^q & \text{by \DEF{5.6.7}}
        \end{align*}
    \item[\(\Longleftarrow\)]
        \begin{align*}
                     & x^q > y^q \\
            \implies & x^{a/b} > y^{a/b} \\
            \implies & (x^{1/b})^a > (y^{1/b})^a & \text{by \DEF{5.6.7}} \\
            \implies & ((x^{1/b})^a)^{1/a} > ((y^{1/b})^a)^{1/a} & \text{\(a\)'s positive int,} \\
                     &                                           & \text{by \LEM{5.6.6}(d)'s \(\Longrightarrow\) direction} \\
            \implies & x^{1/b} = ((x^{1/b})^a)^{1/a} > ((y^{1/b})^a)^{1/a} = y^{1/b} & \text{by \LEM{5.6.6}(b)} \\
            \implies & x^{1/b} > y^{1/b} & \text{simplify} \\
            \implies & x > y & \text{\(b\)'s positive int,} \\
                     &       & \text{by \LEM{5.6.6}(d)'s \(\Longleftarrow\) direction}
        \end{align*}
    \end{itemize}
\item
    Let \(q = a/b, r = c/d\) for some integers \(a, b\) and for some positive integers \(b, d\).
    \begin{itemize}
    \item
        Suppose \(x > 1\).

        Suppose \(x^q > x^r\), we have to show \(q > r\).
        Then
        \begin{align*}
                     & x^q > x^r \\
            \implies & x^q x^{-r} > x^r x^{-r} & \text{\(x^{-r}\) is positive by part(a)}\\
            \implies & x^{q - r} > x^{r - r} & \text{by part(b)} \\
            \implies & x^{q - r} > x^0 = 1 & \text{of course} \\
            \implies & x^{a/b - c/d} > 1 & \\
            \implies & x^{(ad - bc)/bd} > 1 & \text{by \DEF{4.2.2}} \\
            \implies & x^{(ad - bc) \X (1/bd)} > 1 \\
            \implies & (x^{ad - bc})^{1/bd} > 1 & \text{by part(b)} \\
            \implies & ((x^{ad - bc})^{1/bd})^{bd} > 1^{bd} = 1 & \text{\(bd > 0\), by part(d)} \\
            \implies & x^{ad - bc} = ((x^{ad - bc})^{1/bd})^{bd} > 1 & \text{\(bd\)'s positive int, with \LEM{5.6.6}(a)} \\
            \implies & x^{ad - bc} > 1 \MAROON{(*)}
        \end{align*}
        Now we claim that \(ad - bc > 0\).
        It's trivial that \(ad - bc \neq 0\).
        If \(ad - bc < 0\), since \(x > 1\), from \PROP{4.3.12}(b), we have \(x^{ad - bc} < 1^{ad - bc} = 1\), which contradicts \MAROON{(*)}.
        And hence
        \begin{align*}
                     & ad - bc > 0 \\
            \implies & (ad - bc)/bd > 0 \X 1/bd = 0 & \text{(\(1/bd > 0\), with \PROP{4.2.9}(e))} \\
            \implies & (a/b - c/d) > 0 & \text{by \DEF{4.2.2}} \\
            \implies & a/b > c/d \\
            \implies & q > r
        \end{align*}
    
        Now suppose \(q > r\), we have to show \(x^q > x^r\).
        Then
        \begin{align*}
                     & q > r \\
            \implies & q - r > 0 \\
            \implies & x^{q - r} > 1^{q - r} & \text{\(q - r > 0 \land x > 1\), with part(d)} \\
            \implies & x^{q - r} > 1^{(ad - bc)/bd} & \text{by \DEF{4.2.2}} \\
            \implies & x^{q - r} > (1^{1/bd})^{ad - bc} & \text{by \DEF{5.6.7}} \\
            \implies & x^{q - r} > 1^{ad - bc} = 1 & \text{\(bd\)'s positive integer, with \LEM{5.6.6}(e)} \\
            \implies & (x^{q - r}) \X x^r > 1 \X x^r = x^r & \text{\(x^r > 0\), by part(a)} \\
            \implies & x^{q - r + r} > x^r & \text{by part(b)} \\
            \implies & x^q > x^r
        \end{align*}
    \item
        Suppose \(x < 1\).

        Suppose \(x^q > x^r\), we have to show \(q < r\).
        Then
        \begin{align*}
                     & x^q < x^r \\
            \implies & x^q x^{-r} < x^r x^{-r} & \text{\(x^{-r}\) is positive by part(a)}\\
            \implies & x^{q - r} < x^{r - r} & \text{by part(b)} \\
            \implies & x^{q - r} < x^0 = 1 & \text{of course} \\
            \implies & x^{a/b - c/d} < 1 & \\
            \implies & x^{(ad - bc)/bd} < 1 & \text{by \DEF{4.2.2}} \\
            \implies & x^{(ad - bc) \X (1/bd)} < 1 \\
            \implies & (x^{ad - bc})^{1/bd} < 1 & \text{by part(b)} \\
            \implies & ((x^{ad - bc})^{1/bd})^{bd} < 1^{bd} = 1 & \text{\(bd > 0\), by part(d)} \\
            \implies & x^{ad - bc} = ((x^{ad - bc})^{1/bd})^{bd} < 1 & \text{\(bd\)'s positive int, with \LEM{5.6.6}(a)} \\
            \implies & x^{ad - bc} < 1 \MAROON{(**)}
        \end{align*}
        Now we claim that \(ad - bc < 0\).
        It's trivial that \(ad - bc \neq 0\).
        If \(ad - bc > 0\), since \(x > 1\), from \PROP{4.3.12}(b), we have \(x^{ad - bc} > 1^{ad - bc} = 1\), which contradicts \MAROON{(**)}.
        And hence
        \begin{align*}
                     & ad - bc < 0 \\
            \implies & (ad - bc)/bd < 0 \X 1/bd = 0 & \text{(\(1/bd > 0\), with \PROP{4.2.9}(e))} \\
            \implies & (a/b - c/d) < 0 & \text{by \DEF{4.2.2}} \\
            \implies & a/b < c/d \\
            \implies & q < r
        \end{align*}
        
        Suppose \(q < r\), we have to show \(x^q > x^r\).
        Then
        \begin{align*}
                     & q < r \\
            \implies & r - q > 0 \\
            \implies & 1^{r - q} > x^{r - q} & \text{\(r - q > 0 \land 1 > x\), with part(d)} \\
            \implies & x^{r - q} < 1^{r - q} \\
            \implies & x^{r - q} < 1^{c/d - a/b} \\
            \implies & x^{r - q} > 1^{(cb - da)/db} & \text{by \DEF{4.2.2}} \\
            \implies & x^{r - q} > (1^{1/db})^{cb - da} & \text{by \DEF{5.6.7}} \\
            \implies & x^{r - q} > 1^{cb - da} = 1 & \text{\(db\)'s positive integer, with \LEM{5.6.6}(e)} \\
            \implies & (x^{r - q}) \X x^q > 1 \X x^q = x^q & \text{\(x^q > 0\), by part(a)} \\
            \implies & x^{r - q + q} > x^q & \text{by part(b)} \\
            \implies & x^r > x^q
        \end{align*}
    \end{itemize}
\item
    Let \(q = a/b\) for some integer \(a\) and positive integer \(b\).
    Then
    \begin{align*}
        (xy)^{q} & = ((xy)^{1/b})^a & \text{by \DEF{5.6.7}} \\
                 & = (x^{1/b} y^{1/b})^a & \text{\(b\)'s positive integer, with \LEM{5.6.6}(f)} \\
                 & = (x^{1/b})^a (y^{1/b})^a & \text{\(a\)'s integer, with \PROP{4.3.12}(a)} \\
                 & = x^q y^q & \text{by \DEF{5.6.7}}
    \end{align*}
\end{enumerate}
\end{proof}

We still have to do \emph{real} exponentiation;
in other words, we still have to define \(x^y\) where \(x > 0\) and \(y\) is a \emph{real} number,
but we will defer that until \SEC{6.7}, once we have formalized the concept of limit.

\exercisesection

\begin{exercise} \label{exercise 5.6.1}
Prove \LEM{5.6.6}.
(Hints: review the proof of \PROP{5.5.12}.
Also, you will find proof by contradiction a useful tool, especially when combined with the trichotomy of order in \PROP{5.4.7} and \PROP{5.4.12}.
The earlier parts of the lemma can be used to prove later parts of the lemma.
With part(e), first show that if \(x > 1\) then \(x^{1/n} > 1\), and if \(x < 1\) then \(x^{1/n} < 1\).)
\end{exercise}

\begin{proof}
See \LEM{5.6.6}.
\end{proof}

\begin{exercise} \label{exercise 5.6.2}
Prove \LEM{5.6.9}.
(Hint: you should rely mainly on \LEM{5.6.6} and on algebra.)
\end{exercise}

\begin{proof}
See \LEM{5.6.9}.
\end{proof}

\begin{exercise} \label{exercise 5.6.3}
If \(x\) is a real number, show that \(\abs{x} = (x^2)^{1/2}\).
\end{exercise}

\begin{proof}
We split the case into \(x \ge 0\), \(x < 0\).
\begin{itemize}
    \item [\(x \ge 0\)]:
        Then by \LEM{5.6.6}(b) we have \(x = (x^2)^{1/2}\).
        But \(\abs{x} = x\), so \(\abs{x} = (x^2)^{1/2}\).
    \item [\(x < 0\)]:
        Then \(-x > 0\) and by \LEM{5.6.6}(b) we have \(-x = ((-x)^2)^{1/2}\).
        But \((-x)^2 = x^2\), so \(-x = (x^2)^{1/2}\).
        But \(\abs{x} = -x\), so \(\abs{x} = (x^2)^{1/2}\).
\end{itemize}
\end{proof}