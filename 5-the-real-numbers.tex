\chapter{The real numbers} \label{ch 5}

\begin{note}
We defined the natural numbers using the five Peano axioms, and postulated that such a number system existed;
this is plausible, since the natural numbers correspond to the very intuitive and fundamental notion of \emph{sequential counting}.
\end{note}

\begin{note}
The symbols \(\SET{N}\), \(\SET{Q}\), and \(\SET{R}\) stand for ``natural'', ``quotient'', and ``real'' respectively.
\(\SET{Z}\) stands for ``Zahlen'', the German word for numbers.
There is also the \emph{complex numbers} \(\SET{C}\), which obviously stands for ``complex''.
\end{note}

\begin{note}
\emph{Formal} means ``having the form of'';
at the beginning of our construction the expression \(a \M b\) did not actually \emph{mean} the difference \(a - b\), since the symbol \(\M\) was meaningless.
It only had the \emph{form} of a difference.
Later on we defined subtraction and verified that the formal difference was equal to the actual difference, so this eventually became a non-issue, and our symbol for formal differencing was discarded.
Somewhat confusingly, this use of the term ``formal'' is unrelated to the notions of a formal argument and an informal argument.
\end{note}

\begin{note}
There is a fundamental area of mathematics where the rational number system \emph{does not suffice} - that of \emph{geometry} (the study of lengths, areas, etc.).
For instance, a right-angled triangle with both sides equal to \(1\) gives a hypotenuse of \(\sqrt{2}\), which is an \emph{irrational} number, i.e., not a rational number; see \PROP{4.4.4}.
Things get even worse when one starts to deal with the sub-field of geometry known as \emph{trigonometry}, when one sees numbers such as \(\pi\) or \(\cos(1)\), which turn out to be in some sense ``even more'' irrational than \(\sqrt{2}\).
(These numbers are known as \emph{transcendental numbers}, but to discuss this further would be far beyond the scope of this text.)
Thus, in order to have a number system which can adequately describe geometry
- or even something as simple as measuring lengths on a line
- one needs to replace the rational number system with the real number system.
\end{note}

\begin{note}
In the constructions of integers and rationals, the task was to introduce one more \emph{algebraic} operation to the number system
- e.g., one can get integers from naturals by introducing subtraction, and get the rationals from the integers by introducing division.
But to get the reals from the rationals is to pass \emph{from a ``discrete'' system to a ``continuous'' one}, and requires the introduction of a somewhat different notion
- that of a \emph{limit}.
\end{note}

\begin{note}
The limit is a concept which on one level is quite intuitive, but to pin down rigorously turns out to be quite difficult.
(Even such great mathematicians as Euler and Newton had difficulty with this concept.
It was only in the nineteenth century that mathematicians such as Cauchy and Dedekind figured out how to deal with limits rigorously.)
\end{note}

\begin{note}
In \SEC{4.4} we explored the ``gaps'' in the rational numbers;
now we shall fill in these gaps using limits to create the real numbers.
The real number system will end up being a lot like the rational numbers, but will have some new operations
- notably that of \emph{supremum}, which can then be used to define limits and thence to everything else that calculus needs.
\end{note}

\begin{note}
The procedure we give here of obtaining the real numbers as the limit of sequences of rational numbers may seem rather complicated.
However, it is in fact an instance of a very general and useful procedure, that of \emph{\href{https://www.wikiwand.com/en/Complete_metric_space}{completing} one metric space to form another}.
\end{note}

\section{Cauchy sequences}

\begin{definition} [Sequences]  \label{def 5.1.1}
Let \(m\) be an integer.
A sequence \((a_n)_{n = m}^{\infty}\) of \emph{rational} numbers is any function from the set \( \{n \in \SET{Z} : n \ge m\} \) to \(\SET{Q}\),
i.e., a mapping which assigns to each integer \(n\) greater than or equal to \(m\), a rational number \(a_n\).
More informally, a sequence \((a_n)_{n = m}^{\infty}\)
of rational numbers is a collection of rationals \(a_m, a_{m + 1}, a_{m + 2},...\)
\end{definition}

\begin{example}
Simple example.
\end{example}

We want to define the \emph{real} numbers as the \emph{``limits''} of sequences of \emph{rational} numbers.
To do so, we have to distinguish which sequences of rationals are \emph{convergent} and which ones are not.
To do this we use the definition of \DEF{4.3.4} \(\varepsilon\)-closeness.

\begin{definition} [\(\varepsilon\)-steadiness] \label{def 5.1.3}
Let \(\varepsilon > 0\).
A sequence \((a_n)_{n = 0}^{\infty}\) is said to be \emph{\(\varepsilon\)-steady} iff each pair \(a_j, a_k\) of sequence elements is \(\varepsilon\)-close \emph{for every} natural number \(j, k\).
In other words, the sequence \(a_0, a_1, a_2,...\) is \(\varepsilon\)-steady iff \(d(a_j, a_k) \le \varepsilon\) for all \(j, k\).
\end{definition}

\begin{remark} \label{remark 5.1.4}
This definition is not standard in the literature;
we will not need it outside of this section;
similarly for the concept of ``eventual \(\varepsilon\)-steadiness'' below.
We have defined \(\varepsilon\)-steadiness for sequences whose index starts at \(0\), but clearly we can make a similar notion for sequences whose indices start from any other number:
a sequence \(a_N, a_{N + 1}, ...\) is \(\varepsilon\)-steady if one has \(d(a_j, a_k) \le \varepsilon\) for all \(j, k \ge N\).
\end{remark}

\begin{example} \label{example 5.1.5}
The sequence \(1, 0, 1, 0, 1,...\) is \(1\)-steady, but is not \(1/2\)-steady.
The sequence \(0.1, 0.01, 0.001, 0.0001,...\) is \(0.1\)-steady, but is not \(0.01\)-steady (why? Because the distance of the first two elements \(d(0.1, 0.01) = 0.09 > 0.01\)).
The sequence 1, 2, 4, 8, 16,... is not \(\varepsilon\)-steady for any \(\varepsilon\) (why? \MAROON{(1)})
The sequence 2, 2, 2, 2,... is \(\varepsilon\)-steady for every \(\varepsilon > 0\).
\end{example}

\begin{proof}
\MAROON{(1)} Given arbitrary \(\varepsilon\), by \PROP{4.4.1} there exists a natural number \(N\) s.t. \(N > \varepsilon\).
And by definition of the sequence, \(a_{N + 1} = 2^{N + 1}\), which \(\ge 2(N + 1)\), since \(N + 1\) must be positive and that satisfies \EXEC{4.3.5}.
And \(2(N + 1) - a_0 = 2N + 2 - a_0 = 2N + 2 - 1 = 2N + 1 > N\), which implies \(2^{N + 1} - a_0 > N\), that is, \(a_{N + 1} - a_0 > N\).
So we can find the pair \(a_{N + 1}, a_1\) s.t. \(d(a_{N + 1}, a_1) > N > \varepsilon\), so by \DEF{5.1.3}, the sequence is not \(\varepsilon\)-steadiness.
Since \(\varepsilon\) is arbitrary, for all \(\varepsilon\) the sequence is not \(\varepsilon\)-steady.
\end{proof}

\begin{note}
The notion of \(\varepsilon\)-steadiness of a sequence is simple, but does not really capture the \emph{limiting} behavior of a sequence, because it is \emph{too sensitive to the initial members} of the sequence.
So we need a more robust notion of steadiness that does not care about the initial members of a sequence.
\end{note}

\begin{definition} [Eventual \(\varepsilon\)-steadiness] \label{def 5.1.6}
Let \(\varepsilon > 0\).
A sequence \((a_n)_{n = 0}^{\infty}\) is said to be \emph{eventually \(\varepsilon\)-steady} iff the sequence \(a_N, a_{N + 1}, a_{N + 2},...\) is \(\varepsilon\)-steady \emph{for some} natural number \(N \ge 0\).
In other words, the sequence \(a_0, a_1, a_2, ...\) is eventually \(\varepsilon\)-steady iff there exists an \(N \ge 0\) such that \(d(a_j, a_k) \le \varepsilon\) for all \(j, k \ge N\).
\end{definition}

\begin{example} \label{example 5.1.7}
\raggedright The sequence \(a_1, a_2, ...\) defined by \(a_n := 1/n\), (that is, the sequence \(1, 1/2, 1/3, 1/4,...\)) is not \(0.1\)-steady, but is eventually \(0.1\)-steady, because the sequence \(a_{10}, a_{11}, a_{12}, ...\) (that is, \(1/10, 1/11, 1/12,...\)) is 0.1-steady.
The sequence \(10, 0, 0, 0, 0, ...\) is not \(\varepsilon\)-steady for any \(\varepsilon\) less than \(10\), but it is eventually \(\varepsilon\)-steady for every \(\varepsilon > 0\) (why? \MAROON{(1)}).
\end{example}

\begin{proof}
\MAROON{(1)} The distance of the first two elements of the sequence is \(d(10, 0) = 10\), so every \(\varepsilon < 10\) is less than \(d(10, 0)\).
But given any \(\varepsilon > 0\), let \(N = 1\), then for every \(j, k \ge N\), \(d(a_i, a_j) = d(0, 0) = 0 < \varepsilon\).
By \DEF{5.1.6}, the sequence is eventually \(\varepsilon\)-steady.
Since \(\varepsilon\) is arbitrary, for all \(\varepsilon > 0\) the sequence is eventual \(\varepsilon\)-steady.
\end{proof}

Now we can finally define the correct notion of what it means for a sequence of rationals to ``want'' to converge.

\begin{definition} [Cauchy sequences] \label{def 5.1.8}
A sequence \((a_n)_{n = 0}^{\infty}\) of \emph{rational} numbers is said to be a \emph{Cauchy sequence} iff for every rational \(\varepsilon > 0\), the sequence \((a_n)_{n = 0}^{\infty}\) is eventually \(\varepsilon\)-steady.
In other words, the sequence \(a_0, a_1, a_2,...\) is a Cauchy sequence iff for every \(\varepsilon > 0\), there exists an \(N \ge 0\) such that \(d(a_j, a_k) \le \varepsilon\) for all \(j, k \ge N\).
\end{definition}

\begin{remark} \label{remark 5.1.9}
At present, the parameter \(\varepsilon\) is restricted to be a positive \emph{rational};
we cannot take \(\varepsilon\) to be an arbitrary positive \emph{real} number, because the real numbers have not yet been constructed.
However, once we do construct the real numbers, we shall see that \DEF{5.1.8} will not change if we require \(\varepsilon\) to be real instead of rational(\PROP{6.1.4}).
In other words, we will eventually prove that
\begin{center}
    (a sequence is eventually \(\varepsilon\)-steady for every \emph{rational} \(\varepsilon > 0\)) if and only if (it is eventually \(\varepsilon'\)-steady for every \emph{real} \(\varepsilon' > 0\)).
\end{center}
(See \PROP{6.1.4}.)
This rather subtle distinction between a rational \(\varepsilon\) and a real \(\varepsilon'\) turns out \emph{not} to be very important in the long run, and the reader is advised not to pay too much attention as to what type of number \(\varepsilon\) should be.
\end{remark}

\begin{example} [\emph{Informal}] \label{example 5.1.10}
Consider the sequence \(1.4, 1.41, 1.414, 1.4142,...\) mentioned earlier.
This sequence is already \(0.1\)-steady.
If one discards the first element \(1.4\), then the remaining sequence \(1.41, 1.414, 1.4142,...\) is now \(0.01\)-steady, which means that the original sequence was eventually \(0.01\)-steady. 
Discarding the next element gives the \(0.001\)-steady sequence \(1.414, 1.4142,...\);
thus the original sequence was eventually \(0.001\)-steady. 
Continuing in this way it seems plausible that this sequence is in fact \(\varepsilon\)-steady for every \(\varepsilon > 0\), which seems to suggest that this is a Cauchy sequence.
However, this discussion is not rigorous for several reasons, for instance we have not precisely defined what this sequence \(1.4, 1.41, 1.414, \textbf{...}\) really is.
An example of a rigorous treatment follows next.
\end{example}

\begin{note}
我不確定課本說我們還沒定義\ \(1.4, 1.41, 1.414, ...\) 的意思是指什麼。也許可能是我們目前有的工具只有數學歸納法? 目前只能用數學歸納法(或遞迴,或者公式,嚴格來說還是遞迴)來定義\ sequence,但\ \(1.4, 1.41, 1.414, ...\) 其實跟數學歸納法無關,所以嚴格來說我們根本沒說清楚他是什麼?
\end{note}

\begin{proposition} \label{prop 5.1.11}
\sloppy The sequence \(a_1, a_2, a_3,...\) defined by \(a_n := 1/n\) (i.e., the sequence \(1, 1/2, 1/3,...\)) is a Cauchy sequence.
\end{proposition}

\begin{note}
Minor: the index is start from \(1\), not from \(0\).
\end{note}

\begin{proof}
We have to show that for every \(\varepsilon > 0\), the sequence \(a_1, a_2,...\) is eventually \(\varepsilon\)-steady.
So let \(\varepsilon > 0\) be \emph{arbitrary}.
We now have to find a number \(N \ge 1\) such that the sequence \(a_N, a_{N + 1},...\) is \(\varepsilon\)-steady.
This means that \(d(a_j, a_k) \le \varepsilon\) for every \(j, k \ge N\), i.e.
\begin{center}
    \(\abs{1 / j - 1 / k} \le \varepsilon\) for every \(j, k \ge N\).
\end{center}
Now since \(j, k \ge N\), we know that
\begin{align*}
             & 0 < 1 / j \le 1 / N \land 0 < 1 / k \le 1 / N \\
    \implies & 0 \le 1 / j \le 1 / N \land 0 \le 1 / k \le 1 / N & \text{in particular by \DEF{4.2.8}} \\
    \implies & 0 \le 1 / j \le 1 / N \land -(1 / N) \le - (1 / k) \le 0 & \text{by \EXEC{4.2.6}} \\
    \implies & -(1 / N) \le 1 / j - 1 / k \le 1 / N & \text{by \PROP{4.2.9}(e)} \\
    \implies & \abs{1 / j - 1 / k} \le 1 / N & \text{by \PROP{4.3.3}(b)}.
\end{align*}
So in order to force \(\abs{1 / j - 1 / k}\) to be less than or equal to \(\varepsilon\), it would be sufficient for \(1/N\) to be less than (or equal to) \(\varepsilon\).
So all we need to do is choose an \(N\) such that \(1 / N\) is less than \(\varepsilon\), or in other words that \(N\) is
greater than \(1/\varepsilon\).
But this can be done thanks to \PROP{4.4.1}.
(\(\varepsilon\) is rational, so \(1/\varepsilon\) is also rational, by \PROP{4.4.1} we can find a natural number \(N > 1/\varepsilon\).)
\end{proof}

\begin{note}
As you can see, verifying from first principles (i.e., without using any of the machinery of limits, etc.) that a sequence is a Cauchy sequence requires some effort, even for a sequence as simple as \(1/n\).
\emph{The part about selecting an \(N\) can be particularly difficult for beginners} - one has to \emph{think in reverse}, working out what conditions on \(N\) would suffice to force the sequence \(a_N, a_{N + 1}, a_{N + 2}, ...\) to be \(\varepsilon\)-steady, and then finding an \(N\) which obeys those conditions.
Later we will develop some limit laws which allow us to determine when a sequence is Cauchy more easily.
\end{note}

\begin{note}
先去想當\ \(N\) 滿足什麼條件時會導致整個\ sequence 是\ \(\varepsilon\)-steady,然後再找出這個\ \(N\)。
\end{note}

We now relate the notion of a Cauchy sequence to another basic notion, that of a \emph{bounded} sequence.

\begin{definition} [Bounded sequences] \label{def 5.1.12}
Let \(M \ge 0\) be rational.
A \emph{finite} sequence \(a_1, a_2, ..., a_n\) is \emph{bounded by} \(M\) iff \(\abs{a_i} \le M\) for all \(1 \le i \le n\).
An infinite sequence \((a_n)_{n = 1}^{\infty}\) is \emph{bounded by} \(M\) iff \(\abs{a_i} \le M\) for all \(i \ge 1\).
A sequence is said to be \emph{bounded} iff it is bounded by \(M\) \emph{for some} rational \(M \ge 0\).
\end{definition}

\begin{example} \label{example 5.1.13}
The finite sequence \(1, -2, 3, -4\) is bounded (in this case, it is bounded by \(4\), or indeed by any \(M\) greater than or equal to 4).
But the infinite sequence \(1, -2, 3, -4, 5, -6,...\) is unbounded. (Can you prove this? Use \PROP{4.4.1}.)
The sequence \(1, -1, 1, -1,...\) is bounded (e.g., by \(1\)), but is \emph{not} a Cauchy sequence(because in particular it's not eventual \(2.1\)-steady).
\end{example}

\begin{proof}
For the sake of contradiction, suppose the infinite sequence \(1, -2, 3, -4, 5, -6,...\) is bounded, i.e. bounded by some rational \(M\).
By \PROP{4.4.1}, we can find a natural number \(N\) s.t. \(N > M\).
WLOG, if we find \(N = 0\) then we redefine \(N := 1\) to make it positive.
Then it's clear that \(\abs{a_N} = N > M\), so by \DEF{5.1.12}, the sequence is not bounded by \(M\), a contradiction.
\end{proof}

\begin{lemma} [Finite sequences are bounded] \label{lem 5.1.14}
Every finite sequence \(a_1, a_2, ..., a_n\) is bounded.
\end{lemma}

\begin{proof}
We prove this by induction on \(n\).
When \(n = 1\) the arbitrary sequence \(a_1\) is clearly bounded, for if we choose \(M := \abs{a_1}\) then clearly we have \(\abs{a_i} \le M\) for all \(1 \le i \le n\).
Now suppose that we have already proved the lemma for some \(n \ge 1\);
that is, any finite sequence with \(n\) elements is bounded;
we now prove it for \(n + 1\), i.e., we prove every sequence with \(n + 1\) elements \(a_1, a_2, ..., a_{n + 1}\) is bounded.
So given arbitrary sequence of \(n + 1\) elements \(a_1, a_2, ..., a_n, a_{n + 1}\).
By the induction hypothesis we know that the first \(n\) elements \(a_1, a_2, ..., a_n\) is bounded by some \(M \le 0\);
in particular, it must be also bounded by \(M + \abs{a_{n + 1}}\) since \(M + \abs{a_{n + 1}} \ge M\).
On the other hand, \(a_{n + 1}\) is also bounded by \(\abs{a_{n + 1}}\), so is also bounded by \(M + \abs{a_{n + 1}}\) because the latter \(\ge\) the former.
Thus \(a_1, a_2, ... , a_n, a_{n + 1}\) is bounded by \(M +  \abs{a_{n + 1}}\), and is hence bounded.
This closes the induction.
\end{proof}

\begin{note}
Note that while this argument shows that every \textbf{finite} sequence is bounded, no matter how long the finite sequence is, it \textbf{does not} say anything about whether an \textbf{infinite} sequence is bounded or not;
infinity is not a natural number.
\end{note}

However, we have
\begin{lemma} [Cauchy sequences are bounded] \label{lem 5.1.15}
Every Cauchy sequence \((a_n)_{n = 1}^{\infty}\) is bounded.
\end{lemma}

\begin{proof}
Let \((a_n)_{n = 1}^{\infty}\) be an arbitrary Cauchy sequence, we have to show it is bounded by some rational \(M \ge 0\).

Now by \DEF{5.1.8}, given any \(\varepsilon > 0\), the sequence is eventually \(\varepsilon\)-steady.
By \DEF{5.1.6}, that means we can find an index \(N \ge 0\) s.t. \(d(a_j, a_k) \le \varepsilon\) for all \(j, k \ge N\) \MAROON{(1)}.
Now we split the sequence as the first \(a_1, ..., a_{N}\) and \(a_{N + 1}, ...\).

By \LEM{5.1.14}, the first part is finite, so it's bounded.

For the second part and \MAROON{(1)}, and in particular for the index \(j\) fixed as \(N + 1\), \(d(a_{N + 1}, a_k) \le \varepsilon\) for all \(k \ge N\) \MAROON{(2)}.
Now let \(M := \abs{a_{N + 1}} + \varepsilon\), we will show that every element \(a_k\) for all \(k \ge N\) is bounded by \(M\), that is, \(\abs{a_k} \le M\).
Note that \(M = \abs{a_{N + 1}} + \abs{\varepsilon}\) \MAROON{(3)} since \(\varepsilon > 0\).
So let \(k\) be arbitrary integer s.t. \(k \ge N\).
Then
\begin{align*}
             & d(a_{N + 1}, a_k) \le \varepsilon & \text{by \MAROON{(2)}} \\
    \implies & \abs{a_{N + 1} - a_k} \le \varepsilon & \text{by \DEF{4.3.2}} \\
    \implies & \abs{a_{N + 1} - a_k} \le \abs{\varepsilon} & \text{since \(\varepsilon > 0\)} \\
    \implies & \abs{a_{N + 1}} + \abs{a_{N + 1} - a_k} \le \abs{a_{N + 1}} + \abs{\varepsilon} & \text{by \PROP{4.2.9}(d)} \\
    \implies & \abs{a_{N + 1}} + \abs{a_{N + 1} - a_k} \le M & \text{by \MAROON{(3)}} \\
    \implies & \abs{-a_{N + 1}} + \abs{a_{N + 1} - a_k} \le M & \text{by \PROP{4.3.3}(d)} \\
    \implies & \abs{(-a_{N + 1}) + (a_{N + 1} - a_k)} \le \abs{-a_{N + 1}} + \abs{a_{N + 1} - a_k} \le M & \text{by \PROP{4.3.3}(b)} \\
    \implies & \abs{(-a_{N + 1}) + (a_{N + 1} - a_k)} \le M & \text{by \PROP{4.2.9}(c), transitive} \\
    \implies & \abs{-a_k} & \text{by algebra}. \\
    \implies & \abs{a_k} \le M & \text{by \PROP{4.3.3}(d)}
\end{align*}
Since \(k\) is arbitrary, \(\abs{a_k} \le M\) for all \(k \ge N\).
So the second part of the sequence is bounded by \(M\) and hence is bounded.

So the whole sequence is bounded.
\end{proof}

\begin{exercise} \label{exercise 5.1.1}
Prove \LEM{5.1.15}.
\end{exercise}

\begin{proof}
See \LEM{5.1.15}.
\end{proof}