\section{Addition}\label{sec 2.2}

\begin{definition} [Addition of natural numbers] \label{def 2.2.1}
Let \(m\) be a natural number. To add zero to \(m\), we \emph{define} \(0 + m := m\).
Now suppose \emph{inductively} that we have defined how to add \(n\) to \(m\), i.e. we have defined \(n + m\).
Then we can add \(n\INC \) to \(m\) by defining \((n\INC) + m\) := \((n + m)\INC\).
\end{definition}
\begin{note}
So,
\begin{align*}
\BLUE{0 + m = m}\\
1 + m = (0\INC) + m = \BLUE{(0 + m)}\INC = \BLUE{m}\INC \implies \GREEN{1 + m = m\INC} \\
2 + m = (1\INC) + m = (\GREEN{1 + m})\INC = \GREEN{(m\INC)}\INC \implies 2 + m = (m\INC)\INC
\end{align*}
and so on.
\end{note}
\begin{note}
注意,我們有定義\ \(0 + m\) 是什麼,\textbf{但我們沒有定義\ \(m + 0\) 是什麼},i.e. 需要證明他們兩個相等。參考\ \LEM{2.2.2}。
\end{note}
\begin{note}
這個定義因為符合數學歸納法(Axiom \ref{axm 2.5}),所以給定一個自然數\ \(m\), 對於所有自然數\ \(n\),我們都對\ \(n + m\) 做了定義
\end{note}

\begin{additional corollary}\label{ac 2.2.1} (see textbook page 24 below)
Using \AXM{2.1}, \AXM{2.2}, and induction (\AXM{2.5}), that the sum of two natural numbers is \emph{again} a natural number.
\end{additional corollary}
\begin{proof}
Let \(m\) be a particular but arbitrarily chosen natural number.

Base case: \(\GREEN{0 + m} = \BLUE{m}\) by \DEF{2.2.1}.
But \(\BLUE{m}\) is a natural number, so \(\GREEN{0 + m}\) is a natural number. 

Inductive hypothesis: suppose \(n + m\) is a natural number, we have to show that \((n\INC) + m\) is a natural number.
By \DEF{2.2.1} \((n\INC) + m = (\BLUE{n + m})\INC\). But by inductive hypothesis, \BLUE{\(n + m\)} is a natural number, so by \AXM{2.2}, \((\BLUE{n + m})\INC\) is a natural number, so \((n\INC) + m\) is a natural number. This closes the induction.
\end{proof}

\begin{note}
\DEF{2.2.1} 就可以讓我們推出所有以前就在使用的加法的規則了,例如交換律/結合律,這些之後都會證明。
\end{note}

\begin{lemma}\label{lem 2.2.2}
For any natural number \(n\), \(n + 0 = n\).
\end{lemma}
\begin{proof} We use induction.

The base case \(0 + \BLUE{0} = \BLUE{0}\) follows since we know(by \DEF{2.2.1}) that \(0 + \BLUE{m} = \BLUE{m}\) for every natural number \(\BLUE{m}\), and \(\BLUE{0}\) is a natural number.

Now suppose inductively that \(n + 0 = n\). We wish to show that \((n\INC)+0 = n\INC\). But by \DEF{2.2.1}, \((n\INC) + 0\) is equal to \((n + 0)\INC\), which is equal to \(n\INC\) since by hypothesis \(n + 0 = n\). This closes the induction.
\end{proof}

\begin{lemma}\label{lem 2.2.3} For any natural numbers \(n\) and \(m\), \(\BLUE{n + (m\INC)} = \GREEN{(n + m)\INC}\)
\end{lemma}
\begin{note}
這個\ lemma 內等號\ LHS 的\ operand 的順序跟\ \DEF{2.2.1} 相反,是\ \(\BLUE{n + (m\INC)}\) 而不是\ \((m\INC) + n\),而我們還沒有證明加法有交換律(事實上這本書會用這個\ Lemma 來證明加法交換律),所以不能直接從\ \DEF{2.2.1} 得知這個\ Lemma。
\end{note}
\begin{proof}
We induct on \(n\) (keeping \(m\) fixed).

Base case: \(n = 0\). In this case we have to prove
\begin{align*}
    & \BLUE{0 + (m\INC)} = (\GREEN{0 + m})\INC & \text{\MAROON{(1)}}
\end{align*}
But by \DEF{2.2.1}, \(\BLUE{0 + (m\INC)} = m\INC\) and \(\GREEN{0 + m} = m\), so both sides of \MAROON{(1)} are equal to \(m\INC\) and are thus equal to each other.

Now we assume inductively that \(n + (m\INC) = (n + m)\INC\); we now have to show that 
\[\BLUE{(n\INC) + (m\INC)} = (\GREEN{(n\INC) + m})\INC.\]
The \BLUE{left-hand side} is \((n + (m\INC))\INC\) by \DEF{2.2.1}, which is equal to \(((n + m)\INC)\INC\) by the inductive hypothesis. Similarly, we have \(\GREEN{(n\INC) + m} =(n + m)\INC\) by the \DEF{2.2.1}, and so the right-hand side is also equal to \(((n + m)\INC)\INC\). Thus both sides are equal to each other, and we have closed the induction.
\end{proof}

\begin{additional corollary} [on page 26 above] \label{ac 2.2.2}
\(n\INC = n +1\)
\begin{note}
我們可從\ \DEF{2.2.1} 得知\ \(1 + n = n\INC\),但不能得知\ \(n + 1 = n\INC\)
\end{note}
\begin{proof}
    \begin{align*}
        n\INC & = n\INC + 0 & \text{by \LEM{2.2.2}} \\
              & = (n + 0)\INC & \text{by \DEF{2.2.1}} \\
              & = n + (0\INC) & \text{by \LEM{2.2.3}} \\
              & = n + 1
    \end{align*}
\end{proof}
\end{additional corollary}

\begin{proposition}[Addition is commutative]\label{prop 2.2.4} For any natural numbers \(n\) and \(m\), \(n + m = m + n\).
\end{proposition}
\begin{proof}
We shall use induction on \(n\) (keeping \(m\) fixed).

First we do the base case \(n = 0\), i.e., we show \(0 + m = m + 0\). By the \DEF{2.2.1}, \(0 + m = m\), while by \LEM{2.2.2}, \(m + 0 = m\). Thus the base case is done.

Now suppose inductively that \(n + m = m + n\), now we have to prove that \(\BLUE{(n\INC) + m} = \GREEN{m +(n\INC)}\) to close the induction. By the \DEF{2.2.1}, \(\BLUE{(n\INC) + m} = (n + m)\INC\). By \LEM{2.2.3}, \(\GREEN{m + (n\INC)} = (m + n)\INC\), but \((m + n)\INC\) is equal to \((n + m)\INC\) by the inductive hypothesis \(n + m = m + n\). Thus \((n\INC) + m = m +(n\INC)\) and we have closed the induction.
\end{proof}

\begin{proposition}[Addition is associative]\label{prop 2.2.5}
For any natural numbers \(a\), \(b\), \(c\), we have \((a + b) + c = a + (b + c)\).
\end{proposition}
\begin{proof}
We induct on \(c\), keep \(a, b\) fixed.

Let \(a\), \(b\) be particular but arbitrarily chosen natural numbers.

Base case: for \(c = 0\), we want to prove \((a + b) + 0 = a + (b + 0))\). For the LHS, \((\BLUE{a + b}) + 0 = \BLUE{a + b}\) by \LEM{2.2.2}. For the RHS, \(a + \GREEN{(b + 0)} = a + \GREEN{b}\) by \LEM{2.2.2}. So both sides equal to \(a + b\) and thus equal to each other.

Inductive hypothesis: suppose \((a + b) + c = a + (b + c)\), we want to prove \(\BLUE{(a + b) + (c\INC)} = \GREEN{a + (b + (c\INC))}\). By \LEM{2.2.3}, LHS = \(\BLUE{(a + b) + (c\INC)} = \MAROON{((a + b) + c)\INC}\). By \LEM{2.2.3}, RHS = \(\GREEN{a + (b + (c\INC))} = a + (b + c)\INC\), which again by \LEM{2.2.3} = \((a + (b + c))\INC\). But by inductive hypothesis, \((a + b) + c = a + (b + c)\), so \((a + (b + c))\INC = \MAROON{((a + b) + c)\INC}\), which equals to LHS. So RHS = LHS, so we closed the induction.
\end{proof}

\begin{note}
Because of this associativity we can write sums such as \(a + b + c\) without having to worry about which order the numbers are being added together.
\end{note}

\begin{proposition}[Cancellation law]\label{prop 2.2.6} Let \(a\), \(b\), \(c\) be natural numbers such that \(a + b = a + c\). Then we have \(b = c\).
\end{proposition}
\begin{note}
我們不能用任何「減法」或者是「負數」的概念來證明自然數加法消去法。事實上這本書就是用加法的消去法來定義「虛擬減法」(virtual subtraction),這是一種「形式上」(formal)的減法,並且會進一步證明這個虛擬減法跟整數減法等價。可參考第四章\ \DEF{4.1.1}。
\end{note}
\begin{proof}
We proof this by induction on \(a\).

Base case: let \(a = 0\), we have to prove \(0 + b = 0 + c\ \implies b = c\). But by \DEF{2.2.1}, \(0 + b = b\), and \(0 + c = c\), so \(b = 0 + b = 0 + c = c\), so \(b = c\).

Inductive hypothesis: suppose \(a + b = a + c \implies b = c\), we have to prove
\[\BLUE{(a\INC) + b} = \GREEN{(a\INC) + c} \implies b = c,\]
i.e. we have to suppose \(\BLUE{(a\INC) + b} = \GREEN{(a\INC) + c}\) and show \(b = c\).
By \DEF{2.2.1}, \BLUE{LHS} = \((a + b)\INC\), \GREEN{RHS} = \((a + c)\INC\), so \((a + b)\INC = (a + c)\INC\), and by contrapositive of \AXM{2.4}, \(a + b = a + c\), which by inductive hypothesis implies \(b = c\). This closes the induction.
\end{proof}

\begin{note}
至此加法的基本規則都證明了,接下來要討論加法跟「正數」的關係。
\end{note}

\begin{definition}[Positive natural numbers] \label{def 2.2.7} A natural number \(n\) is said to be positive if and only if it is not equal to \(0\).
\end{definition}
\begin{note}
在自然數這個系統且定義包含了\ \(0\) 的情況下,「正數」的意思就是「不是\ \(0\) 的自然數」。
\end{note}

\begin{proposition}\label{prop 2.2.8} If \(a\) is a positive natural number and \(b\) is a natural number, then \(a + b\) is positive (and hence \(b + a\) is also, by \PROP{2.2.4}).
\end{proposition}

\begin{proof}
We prove by induction on \(b\). Let \(a\) be a particular but arbitrarily chosen \emph{positive} natural number.

Base case: let \(b = 0\). Then
\begin{align*}
a + b & = a + 0 & \\
      & = a & \text{by \LEM{2.2.2}} \\
\end{align*}
which by definition is a positive natural number.

Inductive hypothesis: Suppose \(a + b\) is positive, we have to prove \(a + (b\INC)\) is positive.
Then by \LEM{2.2.3}, \(a + (b\INC) = (a + b)\INC\). But by inductive hypothesis \(a + b\) is positive, which by \DEF{2.2.7} is not equal to \(0\), so by \AXM{2.3}, \((a + b)\INC\) is also not equal to \(0\), which again by \DEF{2.2.7} is positive. This closes the induction.
\end{proof}

\begin{corollary} \label{corollary 2.2.9}
If \(a\) and \(b\) are natural numbers such that \(a + b = 0\), then \(a = 0\) and \(b = 0\).
\end{corollary}

\begin{proof}
Suppose for the sake of contradiction that \(a + b = 0\) but \(a \neq 0\) or \(a \neq 0\). If \(a \neq 0\), then by \DEF{2.2.7} \(a\) is positive and then by \PROP{2.2.8} \(a + b\) is positive and by \DEF{2.2.7} cannot be \(0\), which contradicts \(a + b = 0\). If \(b \neq 0\), then similarly \(b\) is positive and \(b + a\) is positive, and by \PROP{2.2.4} \(= a + b\), so \(a + b\) is positive, again a contradiction. Thus both \(a\) and \(b\) must be \(0\).
\end{proof}

\begin{lemma}\label{lem 2.2.10}
Let \(a\) be a positive natural number. Then there \emph{exists exactly one} natural number \(b\) such that \(b\INC = a\).
\end{lemma}
\begin{note}
注意這實際上是個\ if-then 陳述,如果前提不成立(i.e. \(a\) 不是\ positive natural number)則會直接\ vacuously true!
\end{note}
\begin{proof}
We prove by induction on \(a\).

Base case: let \(a = 0\). But by \DEF{2.2.7} \(a\) is not positive. So this Lemma is vacuously true.

Inductive hypothesis: suppose \(a\) is a positive natural number and \(b\) is the unique natural number such that \(b\INC = a\). We have to prove that there exists exact one natural number \(b'\) such that \(b'\INC = a\INC\).
Let \(b' = b\INC\). Then by Axiom of Substitution \AXM{a.7.4} \(b' = a\) since \(b\INC = a\), and \(b'\INC = a\INC\) again by substitution, so the existence part is satisfied.
Now we prove the unique part. Suppose there also exists a natural number \(c\) such that \(c\INC = a\INC\). Then by \AXM{2.4}, \(c = a\), and since \(b' = a\), so \(c = b'\), which means \(b'\) is unique. This closed the induction.
\end{proof}

\begin{note}
現在有了加法的定義跟一些性質,以及正數的定義,我們可以來定義自然數的「order」了。
\end{note}

\begin{definition}[Ordering of the natural numbers] \label{def 2.2.11}
Let \(n\) and \(m\) be natural numbers. We say that \(n\) is \emph{greater than or equal to} \(m\), and write \(n \geq m\) or \(m \leq n\), if and only if we have \(n = m + a\) for some natural number \(a\). We say that \(n\) is \emph{strictly greater than} \(m\), and write \(n > m\) or \(m < n\), if and only if \(n \geq m\) and \(n \neq m\).
\end{definition}
\begin{note}
\(n \geq m\) 跟\ \(m \leq n\) 只有符號上的差別,他們在定義上是等價的。另外定義內的\ \(a\) 只需要是\ natural number 即可,不需要是\ ``positive'' natural number''。
\end{note}
\begin{additional corollary} \label{ac 2.2.3}
\(n\INC > n\) for \textbf{any} natural number \(n\),因為\ (1) \(n\INC \neq n\)(否則會違反\ Axiom \ref{axm 2.4}) (2) \(n\INC \geq n\),因為\ \(n\INC = n + 1\)(by \AC{2.2.2}),而\ \(1\) 就是個\ natural number。

所以,這代表沒有最大的自然數。
\end{additional corollary}

\begin{proposition} [Basic properties of order for natural numbers] \label{prop 2.2.12}
Let \(a\), \(b\), \(c\) be natural numbers. Then
    \begin{enumerate}
        \item (Order is reflexive) \(a \geq a\).
        \item (Order is transitive) If \(a \geq b\) and \(b \geq c\), then \(a \geq c\).
        \item (Order is \emph{anti}-symmetric) If \(a \geq b\) and \(b \geq a\),then \(a = b\). 
        \item (Addition preserves order) \(a \geq b\) if and only if \(a + c \geq b + c\). 
        \item \(a < b\) if and only if \(a\INC \leq b\).
        \item \(a < b\) if and only if \(b = a + d\) for some \emph{positive} number \(d\).
    \end{enumerate}
\end{proposition}
\begin{note}
若忘了\ anti-symmetric 是什麼,可以回去翻離散課本講\ relation 的部分。
\end{note}

\begin{proof}{(a)}
Let \(a\) be a natural number. Then by \DEF{2.2.11}, \(\BLUE{a + 0} \geq \GREEN{a}\) because \(0\) is a natural number. But by \LEM{2.2.2}, \(\BLUE{a + 0} = \BLUE{a}\), so by Axiom of Substitution \AXM{a.7.4}, \(\BLUE{a} \geq \GREEN{a}\).
\end{proof}

\begin{proof}{(b)}
Let \(a, b, c\) be natural numbers s.t. \(a \geq b\) and \(b \geq c\). Then by \DEF{2.2.11}, \(\exists\ \text{natural numbers}\ m, n\) such that \(a = b + m\) and \(b = c + n\). So
\begin{align*}
    a & = b + m & \\
      & = (c + n) + m & \\
      & = c + (n + m) & \text{by \PROP{2.2.5}}\\
\end{align*}
So \(a = c + (n + m)\) for some natural number \(n + m\), so by \DEF{2.2.11}, \(a \geq c\)
\end{proof}

\begin{proof}{(c)}
Let \(a, b\) be natural numbers s.t. \(a \geq b\) and \(b \geq a\). Then by \DEF{2.2.11}, \(\exists\ \text{natural numbers}\ m, n\) such that \(a = b + m\) and \(b = a + n\). So
\begin{align*}
    a & = b + m & \\
      & = (a + n) + m & \\
      & = a + (n + m) & \text{by \PROP{2.2.5}}\\
\end{align*}
But by \LEM{2.2.2}, \(a = a + 0\), so \(a + 0 = a + (n + m)\), and by \PROP{2.2.6} (cancellation law), \(0 = n + m\), and by Corollary \ref{corollary 2.2.9}, \(n = 0\) and \(m = 0\). In particular, \(m = 0\), so \(a = b + m = b + 0 = b\).
\end{proof}

\begin{proof}{(d)}
Let \(a, b, c, d\) be natural numbers. Then
\begin{align*}
         & a \geq b \\
    \iff & a = b + d             & \text{by \DEF{2.2.11}} \\
    \iff & a + c = (b + d) + c   & \text{by Axiom of Substitution \AXM{a.7.4}} \\ 
    \iff & a + c = c + (b + d)   & \text{by \PROP{2.2.4}(commutative law)} \\
    \iff & a + c = (c + b) + d   & \text{by \PROP{2.2.5}(associative law)} \\
    \iff & a + c = (b + c) + d   & \text{by commutative law} \\
    \iff & a + c \geq b + c      & \text{by \DEF{2.2.11}}
\end{align*}
\end{proof}

\begin{proof}{(e)}
Let \(a, b\) be natural numbers.

\(\Longrightarrow \): Suppose \(a < b\). Then by \DEF{2.2.11} \(a \leq b\) \BLUE{(1)} and \(a \neq b\) \BLUE{(2)}. \BLUE{(1)} implies \(a + m = b\) \MAROON{(1)} for some natural number \(m\), and with \BLUE{(2)} by \DEF{2.2.7} implies \(m\) is positive (otherwise \(b = a + m = a + 0 = a\) by \LEM{2.2.2}, contradicting \(a \neq b\)). Since \(m\) is positive, by \LEM{2.2.10}, \(m = n\INC\) \MAROON{(2)} for some natural number \(n\). So
\begin{align*}
    b & = a + m & \text{by \MAROON{(1)}}\\
      & = a + n\INC & \text{by \MAROON{(2)}}\\
      & = (a + n)\INC & \text{by \LEM{2.2.3}} \\
      & = a\INC + n & \text{by \DEF{2.2.1}}
\end{align*}
which by \DEF{2.2.11} implies \(a\INC \leq b\).

\( \Longleftarrow \): Suppose \(a\INC \leq b\). Then by \DEF{2.2.11} \(a\INC + m = b\) for some natural number \(m\), And
\begin{align*}
    & \BLUE{a\INC + m} \\
    & = (a + m)\INC & \text{by \DEF{2.2.1}} \\
    & = \BLUE{a + m\INC} & \text{by \LEM{2.2.3}}
\end{align*}
So \(\BLUE{a\INC + m = a + m\INC} = b\). By Axiom \ref{axm 2.3}, \(m\INC \neq 0\), so by \DEF{2.2.7} is positive, so \(a \neq b\). So \(a + m\INC = b\) for some natural number \(m\INC\) and \(a \neq b\), by \DEF{2.2.11}, \(a < b\).
\end{proof}

\begin{proof}{(f)}

\( \Longrightarrow \): This was proved in the ``if part'' of (e); the \(m\) in there meets the condition.

\( \Longleftarrow \): This was proved in the middle of the ``only if part'' of (e); the positive \(m\INC\) derived in the middle can be treated as \(d\) in (f), and the derivation in (e) derives \(a < b\).
\end{proof}

\begin{proposition}[Trichotomy of order for natural numbers] \label{prop 2.2.13}
Let \(a\) and \(b\) be natural numbers. Then exactly one of the following statements is true: \(a < b\), \(a = b\), or \(a > b\).
\end{proposition}
\begin{proof}
The sketch: (1) at most one condition holds, (2) at least one condition holds; this implies exactly one condition holds.

(1) at most one:
    \begin{enumerate}
        \item If \(a < b\) then by \DEF{2.2.11} \(a \neq b\). Now if \(a > b\) then we have \(a < b \land a > b\), which by \PROP{2.2.12}(c) (with some trivial implications i.e. \(a < b \implies a \leq b\)) implies \(a = b\), contradicting \(a \neq b\).
        \item If \(a > b\) then by \DEF{2.2.11} \(a \neq b\). Now if \(a < b\) then we have \(a > b \land a < b\), which by \PROP{2.2.12}(c) (with some trivial implications i.e. \(a < b \implies a \leq b\)) implies \(a = b\), contradicting \(a \neq b\).
        \item If \(a = b\) then by \DEF{2.2.11} \(a \not < b \) and \(b \not < a\).
    \end{enumerate}

(2) at least one: we use induction on \(a\) and keep \(b\) fixed.

Base case: Let \(a = 0\). Then we have \(a = 0 \leq b\), because \(b = \GREEN{b} + 0 = \GREEN{b} + a\), so there exists a natural number \(\GREEN{b}\) s.t. \(b = \GREEN{b} + a\), which by \DEF{2.2.11} implies \(a \leq b\) (\BLUE{This is the first (why?)}). And this implies \(a < b\) or \(a = b\).

Inductive hypothesis: Suppose \(a\) satisfies the ``at least one'' proposition, i.e. at least one of \(a < b\), \(a = b\), or \(a > b\) is true. We have to prove \(a\INC\) satisfies the ``at least one'' proposition.
    \begin{enumerate}
        \item Suppose \(a < b\), then by \PROP{2.2.12}(e), \(a\INC \leq b\), which means \(a = b\) or \(a < b\).
        \item Suppose \(a = b\), then \(a\INC = b\INC\) by Axiom of Substitution \AXM{a.7.4}, and \(b\INC > b\) by \AC{2.2.3}, so \(a\INC > b\) (\BLUE{This is the third (why?)})
        \item Suppose \(a > b\). By \AC{2.2.3}, \(a\INC > a\). So we have \(a > b\) and \(a\INC > a\), which by \PROP{2.2.12}(b) implies \(a\INC \geq b\). Or more precisely \(a\INC > b\) because if \(a\INC = b\) that implies \(a + 1 = b\), which implies \(a < b\), which contradicts \(a > b\). (\BLUE{This is the second (why?)})
    \end{enumerate}
So in either case \(a\INC\) satisfies at least one of the \(a\INC < b\), \(a\INC = b\) or \(a\INC > b\).
\end{proof}

\begin{note}
上面這些\ order 的性質可以讓我們定義一個\ ``stronger'' version of induction''. 把\ stronger 標起來只是因為它只是看起來比較強,但實際上跟數學歸納法等價。
\end{note}

\begin{proposition}[Strong principle of induction] \label{prop 2.2.14}
Let \(m_0\) be a natural number, and let \(P(m)\) be a property pertaining to an arbitrary natural number \(m\). Suppose that \emph{for each} \(m \geq m_0\), \emph{we have the following \RED{implication}} : if \(P(m')\) is true for all natural numbers \(m_0 \leq m' \RED{<}\ m\), then \(P(m)\) is also true. (In particular, this means that \(P(m_0)\) is true, since in this case the hypothesis is vacuously true.) Then we can conclude that \(P(m)\) is true for all natural numbers \(m \geq m_0\).
\end{proposition}
\begin{note}
要仔細看這個\ proposition 是假設有什麼,然後有什麼結論。他實際上是假設對於任何自然數,我們都有另一個假設。所以整個陳述句是\ nested if-then,所以很不好懂。

這個\ proposition 是說對於任一個\ \(\geq m_0\) 的自然數\ \(m\),都滿足一個「包含了\ \(P\) 還有\ \(m_0\) 的\ if then 陳述句」;若這為真,則會\ implies 「對於所有\ \(\geq m_0\) 的自然數\ 
\(m\),\(P(m)\) 都成立」。

另外當\ \(m = m_0\) 時,那個「被包含的\ if then 陳述句」的\ hypothesis 是\ vacuously true,因為該\ hypothesis 是個\ for all statements,但是\ for all 作用的集合此時為空集合。而如果我們又證明了「那個被包含的\ if then 陳述句」是\ true,則\ (1) if then 陳述句是\ true,\ (2) if-part 也是\ true,這代表\ then part 也是\ true

講這件事情的用意是,現在在\ \(m = m_0\) 的情況下,then part 就是\ \(P(m_0)\) is true。而\ \(P(m_0)\) is true 其實就是傳統\ induction 的\ base case。換句話說這代表「整個\ strong induction 自己的\ if-part」內會\ implies base case,所以只要提供了\ strong induction 需要的那種\ if-part 就不需要證明\ base case is true 了。
\end{note}

\begin{remark}\label{remark 2.2.15}
In applications we usually use \PROP{2.2.14} with \(m_0 = 0\) or \(m_0 = 1\).
\end{remark}

\begin{proof}
Let \(m_0\) be a natural number, \(P(m)\) be a property satisfying the \RED{implication} in the Proposition, and we define \(Q(n)\) be the property that \(P(m)\) is true for all \(m_0 \leq m < n\). First we want to prove \(Q(n)\) is true for all natural number \(n\). If this for-all statement is true, then we can conclude \(P(m)\) is true for all natural number \(m \geq m_0\) (see the last step below). It can be proved by induction.

Base case: Let \(n = \RED{0}\). We want to show \(Q(\RED{0})\) is true. That is, we want to show
\begin{center}
    \(P(m)\) is true for all \(m_0 \leq m\ \RED{< 0}\) \MAROON{\ \ \ (1)}.
\end{center}
But we have shown that \(0 \leq m_0\) (in ``first (why?)'' in \PROP{2.2.13}) for any natural number \(m_0\), so there are two cases: \(m_0 = 0\) or \(m_0 > 0\).
\begin{enumerate}
    \item If \(m_0 = 0\), then \MAROON{(1)} is
        \begin{center}
            \(P(m)\) is true for all \(\RED{0} \leq m < 0\),
        \end{center}
        which is vacuously true because the range \(0 \leq m < 0\) is empty(precisely, no natural number \(< 0\), otherwise it will contradict the ``first (why?)'' in \PROP{2.2.13}).
    \item If \(m_0 > 0\), then \MAROON{(1)} is also vacuous because the range \(m_0 \leq m < 0\) is (also trivially) empty.
\end{enumerate}
So in both case, \MAROON{(1)} is true, that is, \(Q(0)\) is true.

Inductive hypothesis: Suppose for some natural number \(n\) (and WLOG \(n \geq m_0\), or we need to prove some vacuous cases again), \(Q(\RED{n})\) is true, that is
\begin{center}
    \(P(m)\) is true for all \(m_0 \leq m < \RED{n}\) \MAROON{\ \ \ (2)},
\end{center}
we have to prove \(Q(\RED{n\INC})\) is true, that is
\begin{center}
    \(P(m)\) is true for all \(m_0 \leq m < \RED{n\INC}\) \MAROON{\ \ \ (3)}.
\end{center}
By \MAROON{(2)} and the \RED{implication} in the Proposition, it can be derived that \(P(n)\) is true. From this fact and \MAROON{(2)} we derive
\begin{center}
    \(P(m)\) is true for all \(m_0 \leq m\ \RED{\leq}\ n\) \MAROON{(4)}.
\end{center}
But by \PROP{2.2.12}(e), this is equivalent to \MAROON{(3)}, that is, \(Q(n\INC)\) is true. Thus we closed the induction and we conclude that \(Q(n)\) is true for all natural number \(n\).

Now we have to prove that ``\(Q(n)\) is true for all natural number \(n\)'' implies ``\(P(m)\) is true for all natural number \(m \geq m_0\)''. So let \(m\) be a particular but arbitrarily chosen natural number \(\geq m_0\). then in particular of the statement ``\(Q(n)\) is true for all natural number \(n\)'', \(Q(m)\) is true, that is, \(P(m')\) is true for all \(m_0 \leq m' < m\). But again by the \RED{implication} in the Proposition, \(P(m)\) is true. Since \(m\) is arbitrarily chosen, this concludes \(P(m)\) is true for all natural number \(m \geq m_0\).
\end{proof}

\exercisesection

\begin{exercise}\label{exercise 2.2.1}
    Prove \PROP{2.2.5}.
\end{exercise}
\begin{proof}
    See \PROP{2.2.5}.
\end{proof}

\begin{exercise}\label{exercise 2.2.2}
    Prove \LEM{2.2.10}.
\end{exercise}
\begin{proof}
    See \LEM{2.2.10}.
\end{proof}

\begin{exercise}\label{exercise 2.2.3}
    Prove \PROP{2.2.12}.
\end{exercise}
\begin{proof}
    See \PROP{2.2.12}.
\end{proof}

\begin{exercise}\label{exercise 2.2.4}
    Justify the three statements marked (why?) in the proof of \PROP{2.2.13}.
\end{exercise}
\begin{proof}
    See \PROP{2.2.13}.
\end{proof}

\begin{exercise}\label{exercise 2.2.5}
    Prove \PROP{2.2.14}.
\end{exercise}
\begin{proof}
    See \PROP{2.2.14}.
\end{proof}

\begin{exercise}[Principle of backwards induction]\label{exercise 2.2.6}
Let \(n\) be a natural number, and let \(P(m)\) be a property pertaining to the natural numbers such that \RED{whenever \(P(m\INC)\) is true, then \(P(m)\) is true}. \emph{Suppose that \(P(n)\) is also true}. Prove that \(P(m)\) is true for all natural numbers \(m \leq n\); this is known as principle of backward induction (Hint: apply induction to the variable \(n\).)
\end{exercise}
\begin{proof}
Let \(n\) be a natural number, and \(P(m)\) be a property of natural numbers holding the \RED{implication} in the exercise's description.
Wanted:
    \begin{center}
    If \(P(n)\) is true, then \(P(m)\) is true \(\forall m \leq n\) \MAROON{\ \ \ (1)}   
    \end{center}

Use induction on \(n\).

Base case: Let \(n = 0\). Then \MAROON{(1)} becomes
    \begin{center}
    If \(P(\RED{0})\) is true, then \(P(m)\) is true \(\forall m \leq \RED{0}\) \MAROON{\ \ \ (2)}
    \end{center}
Now suppose \(P(0)\) is true, then the ``then-part'' of \MAROON{(2)} becomes trivial because the for-all range only has the natural number \(0\), and \(P(0)\) is supposed to be true. So the base case is true.

Inductive hypothesis: Let \(n\) be a natural number. Suppose \MAROON{(1)} is true. We have to prove
    \begin{center}
    If \(P(n\INC)\) is true, then \(P(m)\) is true \(\forall m \leq n\INC\) \MAROON{\ \ \ (3)}   
    \end{center}
So suppose \GREEN{\(P(n\INC)\)} is true. By the \RED{implication}, \(P(n)\) is true. But with \(P(n)\) and the inductive hypothesis, \BLUE{\(P(m)\) is true \(\forall m \leq n\)}. So we have \GREEN{\(P(n\INC)\)} is true and \BLUE{\(P(m)\) is true \(\forall m \leq n\)}, that is, \(P(m)\) is true \(\forall m \leq n\INC\). So we prove \MAROON{(3)}. This closes the induction.
\end{proof}

\begin{exercise} \label{exercise 2.2.7}
Let \(n\) be a natural number, and let \(P(m)\) be a property pertaining to the natural numbers such that \RED{whenever \(P(m)\) is true, \(P(m\INC)\) is true}.  Show that
    \begin{center}
    If \(P(n)\) is true, then \(P(m)\) is true for all \(m \geq n\). \MAROON{\ \ \ (1)}
    \end{center}
This principle is sometimes referred to as the \emph{principle of induction starting from the base case \(n\)}.
\end{exercise}
\begin{proof}
Let \(n\) be a natural number, \(P(m)\) be a property holding the \RED{implication}. We first define \(Q(m) = P(m + n)\). Then it is trivial that     \begin{center}
    whenever \(Q(m)\) is true, \(Q(m\INC)\) is true \MAROON{\ \ \ (2)}
    \end{center}
because of the \RED{implication} of \(P(m)\).
Now suppose \(P(n)\) is true. We have to show the then-part of \MAROON{(1)}.
Then \(P(\RED{0} + n)\) is true since \(n = 0 + n\) and by Axiom of Substitution \AXM{a.7.4} \(P(n) = P(\RED{0} + n)\). But that just implies Q(\RED{0}) is true since we define \(Q(\RED{0}) = P(\RED{0} + n)\).
So we have \(Q(0)\) is true and \MAROON{(2)}, and they are base case and inductive hypothesis of principle of induction, so by \AXM{2.5}, \(Q(m)\) is true for all natural number \(m\), that is, \(P(m + n)\) is true for all natural number \(m + n\), or \(P(m')\) is true for all natural number \(m' \geq n\). (since \(m + n \geq n\) by \DEF{2.2.11}.)
\end{proof}