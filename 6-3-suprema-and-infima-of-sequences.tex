\section{Suprema and Infima of sequences} \label{sec 6.3}

Having defined the notion of a supremum and infimum of sets of reals, we can now also talk about the supremum and infimum \emph{of sequences}.

\begin{definition} [Sup and inf of sequences] \label{def 6.3.1}
Let \((a_n)_{n = m}^{\infty}\) be a sequence of real numbers.
Then we define \(\sup(a_n)_{n = m}^{\infty}\) to be the supremum of the \emph{set} \(\{a_n : n \ge m\}\), and \(\inf(a_n)_{n = m}^{\infty}\) to the infimum of the same set \(\{a_n : n \ge m\}\).
\end{definition}

\begin{remark} \label{remark 6.3.2}
The quantities \(\sup(a_n)_{n = m}^{\infty}\) and \(\inf(a_n)_{n = m}^{\infty}\) are sometimes written as \(\sup_{n \ge m} a_n\) and \(\inf_{n \ge m} a_n\) respectively.
\end{remark}

\begin{example} \label{example 6.3.3}
Let \(a_n := (-1)^n\); thus \((a_n)_{n=1}^{\infty}\) is the sequence
\(-1, 1, -1, 1,...\).
Then the set \(\{a_n : n \ge 1\}\) is just the two-element set \(\{-1, 1\}\), and hence \(\sup(a_n)_{n = m}^{\infty}\) is equal to \(1\).
Similarly \(\inf(a_n)_{n = m}^{\infty}\) is equal to \(-1\).
\end{example}

\begin{example} \label{example 6.3.4}
Let \(a_n := 1/n\); thus \((a_n)_{n = 1}^{\infty}\) is the sequence \(1, 1/2, 1/3,...\).
Then the set \(\{a_n : n \ge 1\}\) is the \emph{countable}(\RED{warning}) set \(\{1, 1/2, 1/3, 1/4,...\}\).
Thus \(\sup(a_n)_{n = m}^{\infty} = 1\) and \(\inf_{n \ge m} a_n\) (\EXEC{6.3.1}).
Notice here that the infimum of the sequence is \emph{not actually a member of} the sequence, though it becomes very close to the sequence \emph{eventually}.
(So it is a little inaccurate to think of the supremum and infimum as the ``largest element \emph{of} the sequence'' and ``smallest element \emph{of} the sequence'' respectively.)
\end{example}

\begin{proof}
The sequence is bounded by \(1\), it's trivial to prove it using induction.
So the corresponding set \(\{a_n : n \ge 1\}\) of the sequence has an upper bound \(1\).
So by \DEF{5.5.10}, the supremum is the least upper bound of the set.
To show that the supremum is equal to \(1\), we can show that any real number \(x < 1\) is not an upper bound of the set.
But it is of course true because \(1\) belongs to the set.

Now we show the infimum of the sequence is \(0\).
Again, it's trivial that each element of the sequence is greater than \(0\) using induction.
So the corresponding set \(\{a_n : n \ge 1\}\) of the sequence has an lower bound \(0\).
To show that the infimum of the set is equal to \(0\), we can show that any real number \(x > 0\) is not a lower bound of the set.
But by \EXEC{5.4.4}, given \(x > 0\), we can find a natural number \(n\) s.t. \(1/n < x\), and that is the value of \(a_n\), so we have found an element \(a_n\) in the set s.t. \(x > a_n\), hence \(x\) is not a lower bound of the set.
\end{proof}

\begin{note}
\RED{Warning}: the word ``countable'' in \EXAMPLE{6.3.4} is nout defined yet.
It's defined in \CH{8}.
\end{note}

\begin{example} \label{example 6.3.5}
Let \(a_n := n\); thus \((a_n)_{n = 1}^{\infty}\) is the sequence \(1, 2, 3, 4,...\).
Then the set \(\{a_n : n \ge 1\}\) is just the positive integers \(\{1, 2, 3, 4,...\}\).
Then \(\sup(a_n)_{n = 1}^{\infty} = +\infty\) and \(\inf(a_n)_{n = 1}^{\infty} = 1\).
\end{example}

As the last example shows, it is possible for the supremum or infimum of a sequence(of \emph{real} numbers, not \emph{extended real} numbers) to be \(+\infty\) or \(-\infty\).

\begin{additional corollary} \label{ac 6.3.1}
However, if a sequence \((a_n)_{n = m}^{\infty}\) is
bounded, say bounded by \(M\), then all the elements \(a_n\) of the sequence lie between \(-M\) and \(M\), so that the set \(\{a_n : n \ge m\}\) has \(M\) as an upper bound and \(-M\) as a lower bound.
Since this set is clearly non-empty, we can thus conclude that the supremum and infimum of a bounded sequence are \emph{real} numbers (i.e., not \(+\infty\) and \(-\infty\)).
\end{additional corollary}

\begin{proposition} [Least upper bound property] \label{prop 6.3.6}
Let \((a_n)_{n = m}^{\infty}\) be a sequence of ``real numbers'', and let \(x\) be the ``extended real number'' \(x := \sup(a_n)_{n = m}^{\infty}\).

\BLUE{(1)} Then we have \(a_n \le x\) for all \(n \ge m\).

\BLUE{(2)} Also, whenever \(M \in \SET{R}^*\) is an upper bound for \(a_n\) (i.e., \(a_n \le M\) for all \(n \ge m\)), we have \(x \le M\).

\BLUE{(3)} Finally, for every ``extended real number'' \(y\) for which \(y < x\), there exists at least one \(n \ge m\) for which \(y < a_n \le x\).
\end{proposition}

\begin{proof}
\BLUE{(1)} Given any integer \(n \ge m\), \(a_n\) is of course belong to the set \(\{a_n : n \ge m\}\), and by \THM{6.2.11}(a), \(a_n\) is less than the supremum of the set, i.e. \(a_n \le \sup(a_n)_{n = m}^{\infty}\), i.e. \(a_n \le x\).

\BLUE{(2)} Now, if \(M \in \SET{R}^*\) is an upper bound of the sequence, then \(M\) is of course an upper bound of the corresponding set \(\{a_n : n \ge m\}\).
And by \THM{6.2.11}(b), the supremum of the set is less than \(M\), i.e. \(\sup(a_n)_{n = m}^{\infty} \le M\), i.e. \(x \le M\).

\BLUE{(3)} For the last statement of the proposition, suppose for the sake of contradiction that there exists an ``extended real number`` \(y\) for which \(y < x\) \MAROON{(1)} and for all integer \(n \ge m\), \(\lnot (y < a_n \le x)\), i.e. \(y \ge a_n \lor a_n > x\).
First \(a_n > x\) is impossible since \(x\) we have shown from \BLUE{(1)} that \(a_n \le x\).
But then we have for all integer \(n \ge m\), \(y \ge a_n\). which implies \(y\) is an upper bound of the sequence.
But by \BLUE{(2)}, that implies \(\sup(a_n)_{n = m}^{\infty} = x \le y\), which contradicts \MAROON{(1)}.
\end{proof}

\begin{remark} \label{remark 6.3.7}
There is a corresponding Proposition for infima, but with all the references to order reversed, e.g., all upper bounds should now be lower bounds, etc.
The proof is exactly the same.

The corresponding Proposition: Let \((a_n)_{n = m}^{\infty}\) be a sequence of ``real numbers'', and let \(x\) be the ``extended real number'' \(x := \inf(a_n)_{n = m}^{\infty}\).

\BLUE{(1)} Then we have \(a_n \ge x\) for all \(n \ge m\).

\BLUE{(2)} Also, whenever \(M \in \SET{R}^*\) is a \emph{lower bound} for \(a_n\) (i.e., \(a_n \ge M\) for all \(n \ge m\)), we have \(x \ge M\).

\BLUE{(3)} Finally, for every ``extended real number`` \(y\) for which \(y > x\), there exists at least one \(n \ge m\) for which \(y > a_n \ge x\).
\end{remark}

\begin{note}
In the previous section we saw that all convergent sequences are bounded(by \CORO{6.1.17}).
It is natural to ask whether the \emph{converse} is true:
are all bounded sequences convergent?
The answer is no;
for instance, the sequence \(1, -1, 1, -1,...\) is bounded, but not Cauchy and hence not convergent(contrapositive of \PROP{6.1.12}).
However, if we make the sequence \textbf{both bounded and \emph{monotone}} (i.e., increasing or decreasing), then it is true that it must converge.
\end{note}

\begin{proposition} [Monotone bounded sequences converge] \label{prop 6.3.8}
Let \((a_n)_{n = m}^{\infty}\) be a sequence of \emph{real} numbers which has some \emph{finite upper bound} \(M \in \SET{R}\), and which is also increasing (i.e., \(a_{n + 1} \ge a_n\) for all \(n \ge m\)).
Then \((a_n)_{n = m}^{\infty}\) is convergent, and in fact
\[
    \lim_{n \toINF} a_n = \sup(a_n)_{n = m}^{\infty} \le M
\]
\end{proposition}

\begin{proof}
First we call the supremum of the sequence \(\alpha := \sup(a_n)_{n = m}^{\infty}\).
Since the sequence has a ``real''(i.e. finite) number \(M\) as upper bound, by \AC{6.3.1}, \(\alpha\) is a real number.
And by \PROP{6.3.6}, \(\alpha \le M\).
Now what we left is to prove the sequence converges to \(\alpha\).
That is, we have to show given arbitrary \(\varE > 0\), there exists an integer \(N \ge m\) s.t. \(\abs{a_n - \alpha} \le \varE\) for all \(n \ge N\), or equivalently, \(-\varE \le a_n - \alpha \le \varE\), or \(\alpha - \varE \le a_n \le \alpha + \varE\), for all \(n \ge N\).

So let arbitrary \(\varE > 0\).
Now of course \(\alpha - \varE < \alpha\).
And by \PROP{6.3.6}, we can find an integer \(N \ge m\) s.t. \(\alpha - \varE < a_N \le \alpha\).
Also \emph{since the sequence is increasing}, for all \(n \ge N\), \(\alpha - \varE < a_N \le a_n\);
in particular \(\alpha - \varE \le a_n\) for all \(n \ge N\) \MAROON{(1)}.
But again by \PROP{6.3.6}, for all \(n \ge N\), \(a_n \le \alpha\) (\(\alpha\) is the supremum of the sequence) \MAROON{(2)}.
So with \MAROON{(1)(2)} we have \(\alpha - \varE \le a_n \le \alpha\) for all \(n \ge N\), which of course implies \(\alpha - \varE \le a_n \le \alpha + \varE\) for all \(n \ge N\), as desired.
So the sequence converges to \(\alpha\).
\end{proof}

\begin{note}
One can similarly prove that if a sequence \((a_n)_{n = m}^{\infty}\) is bounded \emph{below} and \emph{decreasing} (i.e., \(a_{n + 1} \le a_n)\), then it is convergent, and that the limit is equal to the infimum.
\end{note}

\begin{additional corollary} \label{ac 6.3.2}
Let \((a_n)_{n = m}^{\infty}\) be a sequence of \emph{real} numbers which has some \emph{finite lower bound} \(M \in \SET{R}\), and which is also decreasing (i.e., \(a_{n + 1} \leq a_n\) for all \(n \ge m\)).
Then \((a_n)_{n = m}^\infty\) is convergent, and in fact
\[
    \lim_{n \toINF} a_n = \inf(a_n)_{n = m}^\infty \ge M.
\]
\end{additional corollary}

\begin{proof} (This proof is just modified from the proof of \PROP{6.3.8}.)
First we call the infimum of the sequence \(\beta := \inf(a_n)_{n = m}^{\infty}\).
Since the sequence has a ``real''(i.e. finite) number \(M\) as lower bound, by \AC{6.3.1}, \(\beta\) is also a real number.
And by \RMK{6.3.7}, \(\beta \ge M\).
Now what we left is to prove the sequence converges to \(\beta\).
That is, we have to show given arbitrary \(\varE > 0\), there exists an integer \(N \ge m\) s.t. \(\abs{a_n - \beta} \le \varE\) for all \(n \ge N\), or equivalently, \(-\varE \le a_n - \beta \le \varE\), or \(\beta - \varE \le a_n \le \beta + \varE\), for all \(n \ge N\).

So let arbitrary \(\varE > 0\).
Now of course \(\beta + \varE > \beta\).
And by \RMK{6.3.7}, we can find an integer \(N \ge m\) s.t. \(\beta + \varE > a_N \ge \beta\).
Also \emph{since the sequence is decreasing}, for all \(n \ge N\), \(\beta + \varE > a_N \ge a_n\);
in particular \(\beta + \varE \ge a_n\) for all \(n \ge N\) \MAROON{(1)}.
But again by \RMK{6.3.7}, for all \(n \ge N\), \(a_n \ge \beta\) (\(\beta\) is the infimum of the sequence) \MAROON{(2)}.
So with \MAROON{(1)(2)} we have \(\beta + \varE \ge a_n \ge \beta\) for all \(n \ge N\), which of course implies \(\beta + \varE \ge a_n \ge \beta - \varE\) for all \(n \ge N\), as desired.
So the sequence converges to \(\beta\).
\end{proof}

\begin{note}
A sequence is said to be \emph{monotone} if it is either increasing or decreasing.
From \PROP{6.3.8} and \CORO{6.1.17} we see that a
monotone sequence converges if and only if it is bounded.
(Given monotone sequence, if it converges, then by \CORO{6.1.17}, is bounded(in fact that corollary does not care if it's monotone); if it is bounded, then by \PROP{6.3.8}, it converges.)
\end{note}

\begin{example} \label{example 6.3.9}
The sequence \(3, 3.1, 3.14, 3.141, 3.1415,...\) is increasing, and is bounded above by \(4\).
Hence by \PROP{6.3.8} it must have a limit, which is a real number less than or equal to \(4\).
\end{example}

\begin{note}
\PROP{6.3.8} asserts that the limit of a monotone sequence exists, but does not directly say what that limit is.
Nevertheless, with a little extra work one can often find the limit once one is given that the limit does exist.
\end{note}

\begin{proposition} \label{prop 6.3.10}
Let \(0 < x < 1\).
Then we have \(\lim_{n \toINF} x^n = 0\).
\end{proposition}

\begin{proof}
Since \(0 < x < 1\), one can show that the sequence \((x^n)_{n = 1}^{\infty}\) is decreasing (because \(1 < x \implies 1 \X x^n < x \X x^n \implies x^n < x^{n + 1}\)).
On the other hand, the sequence \((x_n)_{n = 1}^{\infty}\) has a lower bound of 0. (trivial by induction)
Thus by \AC{6.3.2} (infima version of \PROP{6.3.8}) the sequence \((x_n)_{n = 1}^{\infty}\) converges to some limit \(L\).
Since \(x^{n + 1} = x \X x^n\), we thus see from the limit laws (\THM{6.1.19}) that
\begin{align*}
             & \lim_{n \toINF} x^n = L \\
    \implies & x \lim_{n \toINF} x^n = x L & \text{by algebra} \\
    \implies & \lim_{n \toINF} x \X x^n = x L & \text{by \THM{6.1.19}(c)} \\
    \implies & \lim_{n \toINF} x^{n + 1} = x L
\end{align*}
So \((x^{n + 1})_{n = 1}^{\infty}\) converges to \(xL\).
But the sequence \((x^{n + 1})_{n = 1}^{\infty}\) is just the sequence \((x^n)_{n = 2}^{\infty}\) shifted by one, and so they must have the same limits
(by \EXEC{6.1.3}, \((x^{n + 1})_{n = 1}^{\infty}\) and \((x^{n + 1})_{n = 2}^{\infty}\) have the same limit;
and by \EXEC{6.1.4}, \((x^{n + 1})_{n = 2}^{\infty}\) and \((x^n)_{n = 2}^{\infty}\) have the same limit).
But \((x^n)_{n = 2}^{\infty}\) and \((x^n)_{n = 1}^{\infty}\) also have the same limit(by \EXEC{6.1.3}).
So all in all, that implies the limit of \((x^{n + 1})_{n = 1}^{\infty}\), which is \(xL\), is equal to the limit of \((x^n)_{n = 1}^{\infty}\), which is \(L\).
So \(xL = L\), or \(L(x - 1) = 0\).
Since \(x \neq 1\), \(x - 1 \ne 0\), so \(L\) must be \(0\).
Thus \((x_n)_{n = 1}^{\infty}\) converges to \(0\).
\end{proof}

\exercisesection

\begin{exercise} \label{exercise 6.3.1}
Verify the claim in \EXAMPLE{6.3.4}.
\end{exercise}

\begin{proof}
See \EXAMPLE{6.3.4}.
\end{proof}

\begin{exercise} \label{exercise 6.3.2}
Prove \PROP{6.3.6}. (Hint: use \THM{6.2.11}.)
\end{exercise}

\begin{proof}
See \PROP{6.3.6}.
\end{proof}

\begin{exercise} \label{exercise 6.3.3}
Prove \PROP{6.3.8}.
(Hint: use Proposition 6.3.6, \emph{together with} the assumption that an is \emph{increasing}, to show that an converges to \(\sup(a_n)_{n = m}^{\infty}\).)
\end{exercise}

\begin{proof}
See \PROP{6.3.8}.
\end{proof}

\begin{exercise} \label{exercise 6.3.4}
Explain why \PROP{6.3.10} fails when \(x > 1\).
In fact, show that the sequence \((x_n)_{n = 1}^{\infty}\) diverges when \(x > 1\).
(Hint: prove by contradiction and use the identity \((1/x)^n x^n = 1\) and the limit laws in \THM{6.1.19}.)
Compare this with the argument in \EXAMPLE{1.2.3}; can you now explain the flaws in the reasoning in that example?
\end{exercise}

\begin{proof}
Suppose \((x^n)_{n = 1}^{\infty}\) converges when \(x > 1\) (and hence \(0 < 1/x < 1\)).
Let it converge to \(L\).
Then
\begin{align*}
    1 & = \lim_{n \toINF} 1 & \text{of course} \\
      & = \lim_{n \toINF} (1/x)^n x^n & \text{by the identity} \\
      & = (\lim_{n \toINF} (1/x)^n) \X (\lim_{n \toINF} x^n) & \text{by \THM{6.1.19}(b)} \\
      & = 0 \X (\lim_{n \toINF} x^n) & \text{since \(0 < 1/x < 1\) and by \PROP{6.3.10}} \\
      & = 0 \X L \\
      & = 0,
\end{align*}
which is impossible.

The \EXAMPLE{1.2.3} cannot assume the equation \emph{for all} real number \(x\).
It can assume the equation for \(0 < x < 1\), by \PROP{6.3.10}.
\end{proof}