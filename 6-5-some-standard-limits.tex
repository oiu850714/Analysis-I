\section{Some standard limits} \label{sec 6.5}

\begin{note}
\RED{WARNING}: TODO: some steps need to refer to lemmas in \SEC{5.6}.
\end{note}

Armed now with the limit laws(\THM{6.1.19}) and the squeeze test(\CORO{6.4.14}), we can now compute a large number of limits.

A particularly simple limit is that
\begin{additional corollary} \label{ac 6.5.1}
The constant sequence \(c, c, c, c, ...\) have
\begin{center}
    \(\lim_{n \toINF} c = c\) for any constant \(c\).
\end{center}
(But this statement does not require limit law and squeeze test :).
In fact I have already somewhat used this fact before this section.)
\end{additional corollary}

\begin{proof}
Let \((a_n)_{n = 1}^{\infty}\) be a constant sequence where \(a_n = c\) for all \(n \ge 1\).
Then given arbitrary \(\varE > 0\), we can find \(N = 1\) s.t. \(\abs{a_n - c} = \abs{c - c} = \abs{0} = 0 < \varE\) for all \(n \ge N\).
So by \DEF{6.1.5}, the sequence converges to \(c\).
\end{proof}

In \PROP{6.1.11}, we proved that \(\lim_{n \toINF} 1/n = 0\).
This now implies

\begin{corollary} \label{corollary 6.5.1}
We have \(\lim_{n \toINF} 1/n^{1/k} = 0\) for every integer \(k \ge 1\).
\end{corollary}

\begin{proof}
From \LEM{5.6.6}(e) we know that \(1/n^{1/k}\) is a \emph{decreasing} function of \(n\), while being \emph{bounded below} by \(0\) (by \LEM{5.6.6}(c)).
By \PROP{6.3.8} (for decreasing sequences instead of increasing sequences) we thus know that this sequence converges to some limit \(L \ge 0\):
\[
    L = \lim_{n \toINF} 1/n^{1/k}.
\]
Raising this to the \(k^{th}\) power and using the limit laws (or more precisely, \THM{6.1.19}(b) and induction), we obtain
\[
    L^k = \lim_{n \toINF} 1/n.
\]
By \PROP{6.1.11} we thus have \(L^k = 0\);
but this means that \(L\) cannot be positive (else \(L^k\) would be positive, by \LEM{5.6.6}(c)), so \(L = 0\), and we are done.
\end{proof}

Some other basic limits:
\begin{lemma} \label{lem 6.5.2}
Let \(x\) be a real number.
Then the limit \(\lim_{n \toINF} x^n\) exists and is equal to zero when \(\abs{x} < 1\),
exists and is equal to \(1\) when \(x = 1\),
and diverges when \(x = -1\) or when \(\abs{x} > 1\).
\end{lemma}

\begin{proof}
First we have \(-\abs{x}^n \le x^n \le \abs{x}^n\) for all positive integer \(n\) \MAROON{(1)}.
This can be proved by \(-\abs{x} \le x \le \abs{x}\) and induction.

Suppose \(\abs{x} < 1\) (that is, \(-1 < x < 1\)); in particular \(0 \le \abs{x} < 1\).
Then by \PROP{6.3.10}, \(\lim_{n \toINF} \abs{x}^n = 0\).
Also, by \THM{6.1.19}(c), \(\lim_{n \toINF} -\abs{x}^n = -1 \X \lim_{n \toINF} \abs{x}^n = -1 \X 0 = 0\).
So by \CORO{6.4.14}, squeeze test, and \MAROON{(1)}, we have \(\lim_{n \toINF} x^n = 0\).

Now suppose \(x = 1\).
Then we have \(x^n = 1\), a \emph{constant}, for all \(n \ge 1\), by induction.
So by \AC{6.5.1}, \(\lim_{n \toINF} x^n = 1\).

Now suppose \(x = -1\) or \(\abs{x} > 1\) (that is, \(x \le -1\) or \(x > 1\)).
If \(x = -1\), then by \EXAMPLE{6.1.13}, \(\lim_{n \toINF} x^n\) does not exist.
If \(\abs{x} > 1\), we split into two cases:
\begin{itemize}
    \item \(x > 1\): Then by \EXEC{6.3.4}, \((x^n)_{n = 1}^{\infty}\) diverges, so \(\lim_{n \toINF} x^n\) does not exist.
    \item \(x < -1\): Then trivially the sequence is not eventually \(2\)-steady, so is not Cauchy, hence by \THM{6.4.18} \(\lim_{n \toINF} x^n\) does not exist.
\end{itemize}
\end{proof}

\begin{lemma} \label{lem 6.5.3}
For any \(x > 0\), we have \(\lim_{n \toINF} x^{1/n} = 1\).
\end{lemma}

\begin{proof}
We first show that for every \(\varE > 0\) and every real number \(M > 0\), there exists an \(n\) such that \(M^{1/n} \le 1 + \varE\) \BLUE{(1)}.
Since \(1 < 1 + \varE\), we have \(\frac1{1 + \varE} < 1\), and by \LEM{6.5.2}, \(\lim_{n \toINF} (\frac1{1 + \varE})^n = 0\).
But since \((\frac1{1 + \varE})^n = \frac1{(1 + \varE)^n}\), we have \(\lim_{n \toINF} \frac1{(1 + \varE)^n} = 0\).

Now we defined sequence \((a_n)_{n = 1}^{\infty}\) s.t. \(a_n = \frac1{(1 + \varE)^n}\).
Then it's trivial that \(\inf(a_n)_{n = 1}^{\infty} = 0\).
Since \(M > 0\), \(1/M > 0 = \inf(a_n)_{n = 1}^{\infty}\), and by \RMK{6.3.7}(3), there exists \(n \ge 1\) s.t. \(1/M > a_n\).
That is \(1/M > \frac1{(1 + \varE)^n}\).
And that implies \(M < (1 + \varE)^n\).
And by \LEM{5.6.9}(d), we have \(M^{1/n} < ((1 + \varE)^n)^{1/n}\), which by \LEM{5.6.6}(b) is equal to \(1 + \varE\), as desired.

Now we prove the lemma by splitting into cases:
\begin{itemize}
\item \(x \ge 1\):
    Let arbitrary \(\varE > 0\).
    We have to find \(n \ge 1\) s.t. \(\abs{x^{1/n} - 1} \le \varE\), or \(1 - \varE \le x^{1/n} \le 1 + \varE\).
    But by \BLUE{(1)}, we can find \(n \ge 1\) s.t. \(x^{1/n} \le 1 + \varE\).
    Also, since \(x \ge 1\),
    \begin{align*}
                 & x \ge 1 \\
        \implies & x^{1/n} \ge 1^{1/n} = 1 & \text{by \LEM{5.6.6}(d)(e)} \\
        \implies & x^{1/n} \ge 1 - \varE.
    \end{align*}
    So we have found \(n \ge 1\) s.t. \(1 - \varE \le x^{1/n} \le 1 + \varE\), as desired.
\item \(0 < x < 1\):
    Then \(1/x > 1\), so by the previous case we have \(\lim_{n \toINF} (1/x)^{1/n} = 1\),
    which implies \(\lim_{n \toINF} (x^{-1})^{1/n} = 1\), which by \LEM{5.6.9}(b) implies \(\lim_{n \toINF} x^{-1/n} = 1\).
    But by \THM{6.1.19}(e), we have \(\lim_{n \toINF} (x^{-1/n})^{-1} = 1^{-1} = 1\), which by \LEM{5.6.9}(b) implies \(\lim_{n \toINF} x^{1/n} = 1\), as desired.
\end{itemize}
\end{proof}

\begin{note}
We will derive a few more standard limits later on, once we develop the root and ratio tests(\SEC{7.5}) for series and for sequences.
\end{note}

\exercisesection

\begin{exercise} \label{exercise 6.5.1}
\BLUE{(1)} Show that \(\lim_{n \toINF} 1/n^q = 0\) for any \emph{rational} \(q > 0\).
(Hint: use \CORO{6.5.1} and the limit laws, \THM{6.1.19}.)

\BLUE{(2)} Conclude that the limit \(\lim_{n \toINF} n^q\) does not exist.
(Hint: argue by contradiction using \THM{6.1.19}(e).)
\end{exercise}

\begin{proof}
First we show \BLUE{(1)}.
Let arbitrary rational \(q = a/b > 0\), where \(a, b\) are positive integers.
Then by \CORO{6.5.1} we have \(\lim_{n \toINF} 1/n^{1/b} = 0\).
And with limit laws \THM{6.1.19}(b) and induction, we have \(\lim_{n \toINF} (\frac1{n^{1/b}})^a = 0^a = 0\),
which implies \(\lim_{n \toINF} \frac{1^a}{(n^{1/b})^a} = 0\),
which implies \(\lim_{n \toINF} \frac1{(n^{1/b})^a} = 0\),
which by \DEF{5.6.7} implies \(\lim_{n \toINF} \frac1{n^q} = 0\), as desired.

Now we show \BLUE{(2)}.
Suppose for sake of contradiction that \(\lim_{n \toINF} n^q\) exists and equals to \(L\).
Then
\begin{align*}
    0 & = L \X 0 \\
      & = (\lim_{n \toINF} n^q) \X 0 \\
      & = (\lim_{n \toINF} n^q) \X (\lim_{n \toINF} 1/n^q) & \text{by \BLUE{(1)}} \\
      & = \lim_{n \toINF} n^q \X 1/n^q & \text{by \THM{6.1.19}(b)} \\
      & = \lim_{n \toINF} n^q \X (1/n)^q & \text{of course...} \\
      & = \lim_{n \toINF} (n(1/n))^q & \text{by \LEM{5.6.9}(f)} \\
      & = \lim_{n \toINF} 1^q \\
      & = 1^q & \text{by \AC{6.5.1}} \\
      & = 1,
\end{align*}
which is impossible.
\end{proof}

\begin{exercise} \label{exercise 6.5.2}
Prove \LEM{6.5.2}.
(Hint: use \PROP{6.3.10}, \EXEC{6.3.4}, and the squeeze test, \CORO{6.4.14}.)
\end{exercise}

\begin{proof}
See \LEM{6.5.2}.
\end{proof}

\begin{exercise} \label{exercise 6.5.3}
Prove \LEM{6.5.3}.
(Hint: you may need to treat the cases \(x \ge 1\) and \(x < 1\) separately.
You might wish to first use \LEM{6.5.2} to prove the preliminary result that for every \(\varE > 0\) and every real number \(M > 0\), there exists an \(n\) such that \(M^{1/n} \le 1 + \varE\).)
\end{exercise}

\begin{proof}
See \LEM{6.5.3}.
\end{proof}